\subsection{Введение в комплы.}

\subsubsection{Как задаем компклексные?}

\deff{Множество комплексных чисел} - линейное пространство $\mathbb{R}^2$ с евклидовой нормой.

При этом $x=Re\,z$ - вещественная часть числа, а $y=Im\,z$ - мнимая часть.

При $x=0$ число становится чисто мнимым.

При $y=0$ число можно отождествлять с вещественным числом $x$.

Получаем первый вариант записи комплексных чисел - \deff{Декартову форму}: $$(x; y)=z\in\mathbb{C}; x, y\in\mathbb{R}$$.

\subsubsection{Модуль комлексного.}
Евклидову норму $|z|=||(x;y)||_2=\sqrt{x^2 + y^2}$ называют модулем комплексного числа.


\subsubsection{Различные формы записи.}
\begin{enumerate}
    \item Декартова - разбирали выше.
    \item Алгебраическую форма записи: $z=(x;y)=x+iy$.
    \item Тригонометрическая:
   Введем полярную систему координат с центром, совпадающим с центром д.с.к. и осью вдоль оси $Re\,z$. Тогда для каждого ненулевого комплексного числа получим $r$ и $\varphi$. Тогда триг. запись ---  $r (\cos \varphi + i \sin \varphi)$
   \item Экспоненциальная: $z=r(\cos{\varphi}+i\sin{\varphi})=re^{i\varphi}$
\end{enumerate}

\subsection{Операции сложения, умножения}
\deff{Сложение/вычитание} -- аналогично сложению/вычитанию векторов
          $$(x_1 + iy_1) + (x_2 + iy_2) = (x_1 + x_2) + i(y_1 + y_2)$$
          $$(x_1 + iy_1) - (x_2 + iy_2) = (x_1 - x_2) + i(y_1 - y_2)$$
\deff{Умножение} --
         $ (x_1 + iy_1) * (x_2 + iy_2) = (x_1x_2 - y_1y_2) + i(x_1y_2 + x_2y_1)$
          Распишем в тригонометрической форме перемножение двух комплексных чисел:
          $$r_1r_2(\cos{(\varphi_1 + \varphi_2)} + i\sin{(\varphi_1 + \varphi_2)})$$
\subsection{Операция сопряжения и деления. Комплексные = поле.}

\deff{Сопряжение} --- для всех комплексных чисел $z=x+iy$ существует  сопряжённое ему $\overline{z}=x-iy$. Тривиальные свойства:
          \begin{itemize}
              \item $\overline{\overline{z}} = z$
              \item $z=\overline{z} \Leftrightarrow (x+iy)=(x-iy) \Leftrightarrow y=0 \Leftrightarrow z\in \mathbb{R}$
              \item $z\overline{z}=(x + iy)(x - iy)=(x^2 + y^2)=|z|^2$
              \item $z+\overline{z}=(x + iy) + (x - iy)= 2x = 2 \cdot Re\,z$
              \item $z-\overline{z}=(x + iy) - (x - iy)= 2iy = 2i \cdot Im\,z$
          \end{itemize}
\deff{Обратное} --- $z^{-1}=\cfrac{\overline{z}}{|z|^2}$

\deff{Деление} --- $\cfrac{z_1}{z_2} = z_1 \cdot z_2^{-1}$.

Ну а понять почему комплексное поле - изи. 

\subsection{Свойства экспоненты чисто мнимого числа. Формулы Эйлера, Муавра, корня n – ой степени.}

\subsubsection{Свойства экспоненты чисто мнимого чисто мнимого числа + Формула Эйлера.}

Сделаем заявление, в которое поверим и в дальнейшем будем активно использовать:
$$e^{i\varphi}=\cos{\varphi} + i\sin{\varphi}; \varphi\in\mathbb{R}$$
Свойства:
\begin{enumerate}
    \item $e^{i*2\pi k} = 1; k\in\mathbb{Z}$
    \item $e^{i(\varphi + 2\pi k)} = e^{i\varphi}; k\in\mathbb{Z}$
    \item $e^{i(\varphi_1 + \varphi_2)} = e^{i\varphi_1} \cdot e^{i\varphi_2}$
    \item $e^{-i\varphi} = \frac{1}{e^{i\varphi}} = \overline{e^{i\varphi}}$
    \item $|e^{i\varphi}| = 1$
    \item $e^{i\varphi \cdot n} = (e^{i\varphi})^n; n\in\mathbb{Z}$
    \item \textbf{Формулы Эйлера:}
          $$\frac{e^{i\varphi}+e^{-i\varphi}}{2}=\cos{\varphi},  \frac{e^{i\varphi}-e^{-i\varphi}}{2i}=\sin{\varphi}$$
\end{enumerate}

\subsubsection{Формула Муавра.}
 $\forall n\in\mathbb{N}(|z|(\cos\varphi+i\sin\varphi))^n=|z|^n(\cos n\varphi+i\sin n\varphi)$. Очевидно

 \subsubsection{Корень n-ой степени.}

 \deff{Корнем $n$-ной степени} комплексного числа $re^{i\varphi}$ называются числа $\sqrt[n]re^{i\frac{\varphi+2\pi k}n}$ для $k\in[0:n-1]$. Это очевидно выводится из $\sqrt[n]{w} = z$. $w = z^n$

 \subsubsection{Вычисление квадратного корня в алгебраическом виде.}
            \[
            a+bi=(u+vi)^2\Leftrightarrow
            \left\{\begin{aligned}
                a&=u^2-v^2\\
                b&=2uv
            \end{aligned}\right.
            \]
            Заметим, что $a^2+b^2=(u^2+v^2)^2$, а поскольку и $a^2+b^2$, и $u^2+v^2$ неотрицательны, это равенство значит, что $u^2+v^2=\sqrt{a^2+b^2}$. А отсюда и из $a=v^2-v^2$ получаем, что $u^2=\frac{\sqrt{a^2+b^2}+a}2$ и $v^2=\frac{\sqrt{a^2+b^2}-a}2$. Обе дроби неотрицательны, значит из них можно брать арифметические квадратные корни. Осталось лишь понять, как брать знаки при этих корнях. Да очень просто. Смотрим на $b=2uv$, что значит, что при положительном $b$ мы берём оба корня с одинаковыми знаками, а при отрицательном --- с разными.


\subsection{Функции комплексного аргумента: $\exp z$, $\ln z$, $z^w$, $w^z$}
\subsubsection{Экспонента комплексного аргумента.}


\deff{Комплексная экспонента} -- функция $\exp(x+i y)=e^x(\cos y+i\sin y)=e^x\cdot e^{iy}$. Ее обозначают $e^z$.

\begin{enumerate}
    \item $e^{z + 2\pi ki} = e^z$ -- $2\pi i$ периодичность
    \item $|e^z| = e^x=e^{Re\,z}$
    \item $e^{z_1 + z_2} = e^{z_1} \cdot e^{z_2}$
    \item $e^{-z} = \frac{1}{e^z}$
\end{enumerate}
 Аналогично формулам Эйлера введём sin и cos комплексной переменной:
          $$\cos{z}=\frac{e^{iz}+e^{-iz}}{2},\sin{z}=\frac{e^{iz}-e^{-iz}}{2i}$$
\subsubsection{Логарифм комплексного аргумента.}

Пусть $\ln{z}=w=x+iy$, тогда
$$z=|z|e^{i(\arg{z}+2\pi k)}=re^{i\varphi}$$
$$z=e^{w}=e^x e^{iy}$$
Получим, что $|z|=e^x\in\mathbb{R}$, то есть $x=\ln{|z|}$. А $y=\arg{z}+2\pi k$.

Видим, что в формуле присутствует $2\pi k$, что говорит нам о многозначности логарифма комплексного числа.

\subsubsection{Комплексное в степени комплексного числа.}

Пусть $k\in\mathbb{Z}$, $b\in\mathbb{C}$ -- константа

$w=z^b=e^{b\ln_k{z}}$ -- обобщённая степенная функция

$w=b^z=e^{z\ln_k{b}}$ -- обобщённая показательная функция

Заметим, что стандартные свойства натурального логарифма \textbf{не выполняются}. Например $b^{z_1 + z_2} \neq b^{z_1} b^{z_2}$.

