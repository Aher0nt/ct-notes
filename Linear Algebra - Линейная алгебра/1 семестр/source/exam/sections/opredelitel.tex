\subsection{Полилинейные формы.}
\subsubsection{Полилинейные формы.}
\(\dim V = n\) - лин. пространство над полем K

\(f: V \times V \times \ldots \times V \rightarrow K\) (p штук) - называется \uline{\textbf{полилинейной}} формой (функцией), если она линейна по каждой своей координате

\(f(\xi_1,\ldots,\xi_p) = \) число в K.

\(\forall \lambda \in K, \forall \psi, \mu \in V: f(\ldots,\psi + \lambda \mu,\ldots) = f(\ldots, \psi, \ldots) + \lambda f(\ldots,\mu,\ldots)\)

\textbf{Правило/Соглашение Эйнштейна:} \(x^i e_i = \sum\limits_{i=1}^n x e_i\) - меняем обозначение
$\xi_1,\ldots, \xi_p \in V$.

\(\xi_j  = \xi_j^ie_i \leftrightarrow \begin{pmatrix}
    \xi_j^1 \\
    \vdots  \\
    \xi_j^n
\end{pmatrix} \in K^n \)

\(f(\xi_1,\ldots,\xi_p) = f(\xi_1^{i_1}e_{i_1}, \xi_2^{i_2}e_{i_2},\ldots, \xi{i_p}_p e_{i_p}) =  \xi_1^{i_1} \xi_2^{i_2}\ldots \xi_p^{i_p} f(e_{i_1},
\ldots,e_{i_p})\) %TODO: координаты/ компоненты

Чтобы задать полилинейную форму надо задать значение в базисе и все.
\subsubsection{Антисимметричные полилинейные формы.}

Полилинейная форма называется \deff{антисимметричной}, если при подстановке в неё двух одинаковых аргументов, результат будет равен нулю.

\textbf{Утв.} f антисимметрична \(\Leftrightarrow \forall(i,j): f(\ldots,\xi_i,\ldots, \xi_j,\ldots) =- f(\ldots,\xi_j,\ldots, \xi_i,\ldots) \).

\begin{enumerate}
    \item[] \prooff{}
    \(f(\ldots, \xi_i + \xi_j,\ldots, \xi_i + \xi_j,\ldots) = 0\)

Разложим через линейность:

\(f(\ldots, \xi_i,\ldots,  \xi_j,\ldots)  + f(\ldots, \xi_i,\ldots,  \xi_j,\ldots)  = 0\). Откуда уже следует искомое.

В обратную сторону  \(f(\ldots, \xi ,\ldots, \xi_i,\ldots) = - f(\ldots, \xi ,\ldots, \xi_i,\ldots)\), откуда уже следует искомое.  Q.E.D
\end{enumerate}
\subsubsection{Подстановки}

\(\varphi: (1, \ldots,n) \rightarrow (1,\ldots,n)\) подстановка. Удобнее всего показывать стрелочками. Перестановка - образ.

\(\varphi,\psi \) - 2 подстановки. Произведением перестановок назовем образ композиции отображений.

Произведение ассоциативно, но не коммутативно.

Если $\varphi $ - подстановка, то $\varphi^{-1}$ - взаимно однозначная и взаимообратная.

\textbf{Транспозиция} элементов перестановки \(\sigma\) называется подстановка меняющая местами 2 эл-та перестановки: \((i_1,\ldots,i_a,\ldots,i_b,\ldots,i_n)\)
перейдет в \((i_1,\ldots,i_b,\ldots,i_a,\ldots,i_n)\)

Возьму перестановку. Можно привести к тривиальной. \textbf{Алгоритм:} Найдем 1, вставим в начало, найдем 2, вставим на второе место и так далее.

Назову перестановку \textbf{четной} или \textbf{нечетной}, если я привожу к тривиальной за четное или соответственно нечетное количество транспозиций.

\(\varepsilon(\sigma) = \begin{cases}
    0, \sigma - \text{четная} \\
    1, \sigma - \text{нечетная}
\end{cases}\)

\((-1)^{\varepsilon(\sigma)}\) - знак перестановки

\(f(\xi_1,\ldots,\xi_n) = \sum\limits_{\sigma \in S_n} \xi_1^{i_1}\ldots\xi_n^{i_n} \alpha_{i_1\ldots i_n} = a \sum\limits_{\sigma \in S_n} \xi_1^{i_1} \ldots \xi_n^{i_n} (-1)^{\varepsilon(\sigma)}\).

где \(const = a = f(e_1,\ldots, e_n)\).

n - форму, у которой значение на упорядоченном наборе базиса векторов \(e_1,\ldots,e_n\). равно 1 назовем D.

D - n форма, т.к \(D(e_1,\ldots,e_n) = 1: \forall \xi_1 \ldots \xi_n: D(\xi_1,\ldots, \xi_n) = \sum\limits_{\sigma \in S_n} \xi_1^{i_1} \ldots \xi_n^{i_n} (-1)^{\varepsilon(\sigma)}\) = \(\det (\xi_1,\ldots, \xi_n)\) - \uline{\textbf{определитель системы векторов}.}

\textbf{Замечания:}
\begin{enumerate}
    \item D, \(\forall f\) n-форма  \(f = \alpha D\), где $\alpha = f(e_1,\ldots,e_n)$.
    \item форма D существует единственная.
    \item Определение D-формы зависит от базиса, т.кю чтобы её определить должен быть зафиксирован базис.
\end{enumerate}

\subsubsection{Определитель матрицы}

D - n - форма \(D(E_1,\ldots, E_n) =1 \)

\(\forall A_1,\ldots A_n \in K^n \)

\(D(A_1,\ldots,A_n)  = \sum\limits_{\sigma \in S_n} (-1)^{\varepsilon(\sigma)} a_{i_11}\ldots a_{i_nn} = det A\)

\subsection{Определитель числовой матрице. Теорема об определителе транспонированной, формулы для вычисления определителей первого и второго порядка.}

\subsubsection{ Определитель матрицы, вторая формула.}
\(inv(\sigma) = \) число инверсий в перестановке

\textbf{Теорема}:

\begin{enumerate}
    \item \(\varepsilon(\sigma) = \varepsilon(\sigma^{-1})\)
    \item Любая транспозиция элементов может быть получена за нечётное число транспозиций соседних элементов
    \item транспозиция любых двух соседних элементов меняет число инверсий на 1
    \item \((-1)^{\varepsilon(\sigma)}=(-1)^{inv(\sigma)}\)
\end{enumerate}

Доказательство:

\begin{enumerate}
    \item Получим \(\sigma\) из тривиальной перестановки транспозициями. Применим эти транспозиции в другую сторону. Получим обратную перестановку. Их одно и тоже число.
    \item Хотим поменять местами \(i_\alpha\) и \(i_\beta\). приблизим \(i_\beta\) к \(i_\alpha\) \(k\) транспозициями соседних элементов. Поменяем \(i_\alpha\) и \(i_\beta\) местами. Отодвинем \(i_\alpha\) от \(i_\beta\) \(k\) транспозициями. Всего \(2k + 1\) транспозиций.
    \item Пусть перестановка имеет вид \(A, i_\alpha, i_\beta, B\), где \(A\) и \(B\) --- части перестановки. \(i_\alpha\) образует \(m\) инверсий с \(A\), \(i_\beta\) образует \(k\) инверсий с \(B\). Транспозиция  \(i_\alpha\) и \(i_\beta\) или создаст или уничтожит их инверсию и не изменит \(m\) или \(k\).
    \item  Пусть \(\sigma\) четная = нечетное число соседних транспозиций = четное число  соседних транспозиций приводя к тривиал., т.е число инверсий изменилось на четное число. число инверсий в конце 0, а значит изначально 0, а значит \(inv \sigma =\) четное число. Пусть \( \sigma\) - нечетная = нечетное число соседних транспозиций = нечетное число транспозиций приводит  к тривиальной перстановке. Т.е число транспозиций уменьшилось на неч. число единиц, откуда и следует искомое. %TODO: привести в адекватный вид
\end{enumerate}

Вторая формула для \(\det A\): \(\det A = \sum\limits_{\sigma \in S_n} (-1)^{inv(\sigma)} a_{i_1 1}, \ldots, a_{i_n n}\), где \(\sigma = (i_1, \ldots, i_n)\).

\subsubsection{Теорема об определителе транспонированной матрицы. Свойства определителя}

\begin{enumerate}
    \item \(\det A^T = \det A\)

          \(A^T = \begin{pmatrix}
              A_1^T  \\
              \vdots \\
              A_n^T
          \end{pmatrix}\)

          \(\det A^T = \sum\limits_{\sigma \in S_n} (-1) ^{\varepsilon(\sigma)} a_{1 i_1}, \ldots, a_{n,i_n}\), \(\sigma = (i_1, \ldots, i_n) = (\varphi(1), \ldots, \varphi(n)) = \varphi(1, \ldots, n) \Leftrightarrow (\det A^T = \sum\limits_{\sigma \in S_N} (-1) ^{\varepsilon(\sigma) a_{j+1}}\)

          Следствие: \(det A = \sum\limits_{\sigma \in S_n}(-1)^{inv (\sigma)a_{1i_1}\ldots a_{ni_n}}\), для \(\sigma = (i_1\ldots i_n)\)

          Замечание: все свойства, сформулированные для столбцов, верны и для строк.

    \item \(\det(\ldots, \lambda A_i, \ldots) = \lambda \det(\ldots, A_i, \ldots)\), \(\lambda \in K\)

          \(\det(\ldots, A_i + A_j, \ldots, A_k, \ldots) = \det(\ldots, A_i, \ldots, A_k, \ldots) + \det(\ldots, A_j, \ldots, A_k, \ldots)\)

          Доказательство:

          \(\det A  = D(A_1,\ldots, A_n) \) - полилинейная n - форма, откуда все и следует

    \item \(\det ( \ldots0\ldots) = 0 \) - частный случай $\lambda =0$.

    \item \(\det(\ldots, A_i, \ldots, A_j, \ldots) = -\det(\ldots, A_j, \ldots, A_i, \ldots)\)

          \(\det(\ldots, A_i, \ldots, A_i, \ldots) = 0\)

          Доказательство:
          \(\det\) --- антисимметричная

    \item \(\det (\ldots A_i\ldots A_j\ldots)\) = \(\det (\ldots A_i + \lambda A_j\ldots A_j\ldots)\)

          Доказательство:

          \(\det (\ldots A_i + \lambda A_j\ldots A_j\ldots) = \det (\ldots A_i + \ldots A_j\ldots)  + \det (\ldots \lambda A_j\ldots A_j\ldots) = \det (\ldots A_i + \ldots A_j\ldots) + \lambda\cdot 0\) Q.E.D


    \item Определитель ступенчатой (блочно-диагональной) матрицы:

          \(\det
          \begin{pmatrix}
              A^1    & 0      & \ldots & 0      \\
              *      & A^2    & \ldots & 0      \\
              \vdots & \vdots & \ddots & \vdots \\
              *      & *      & \ldots & A^m
          \end{pmatrix} = \prod\limits_{k=1}^m \det A^k\)


          \(A^k = (a_{i_j}^k)\)

          Доказательство:
          \begin{itemize}
              \item База \(m = 2\):
                    \(det\begin{pmatrix}
                        A_1 & 0   \\
                        *   & A_2 \\
                    \end{pmatrix}\)


                    Решим простой случай \(A_1 = 1, A_2 = 1\):

                    \(\det\begin{pmatrix}
                        E_1 & 0   \\
                        *   & E_2 \\
                    \end{pmatrix} =
                    \det\begin{pmatrix}
                        E_1 & 0   \\
                        0   & E_2 \\
                    \end{pmatrix} = \det E = 1\)\


                    Усложним. Пусть у нас теперь только одна из двух матриц единичная(\(E_{k_2}\) - единичная матрица размера \(k \times k\)):


                    \(\det\begin{pmatrix}
                        B & 0       \\
                        * & E_{k_2} \\
                    \end{pmatrix} =
                    \det\begin{pmatrix}
                        B & 0       \\
                        0 & E_{k_2} \\
                    \end{pmatrix} \)
                    \(= f(B_1, \ldots, B_{k_1}) = f(E_1, \ldots, E_{k_1}) \det B = \det B\)

                    \(f\) --- \(k_1\)-форма, значит полилинейная и антисимметричная. (\(f\) - функция, которая для заданной \(B\) находит определитель матрицы)

                    \(f =\alpha D\), \(\alpha = f(e_1, \ldots, e_{k_1})\)

                    \(f(E_1, \ldots, E_{k_1}) = \det\begin{pmatrix}
                        E_{k_1} & 0       \\
                        *       & E_{k_2} \\
                    \end{pmatrix} = 1\)


                    Усложним ещё раз:

                    \(\det\begin{pmatrix}
                        B & 0 \\
                        * & C \\
                    \end{pmatrix} = g(C_1,\ldots,C_{k_2}) = g(E_1,\ldots,E_{k_2})\cdot \det C = \det B \det C\), что следует из того, что \(g\) - полилинейная форма и из прошлого

              \item Индукционный переход
                    Пусть верно для \(m - 1\), тогда докажем, что верно для \(m\):

                    \(\det
                    \begin{pmatrix}
                        A^1    & 0      & \ldots & 0      \\
                        *      & A^2    & \ldots & 0      \\
                        \vdots & \vdots & \ddots & \vdots \\
                        *      & *      & \ldots & A^m
                    \end{pmatrix} = \begin{pmatrix}
                        A & 0   \\
                        * & A^m \\
                    \end{pmatrix} = \det A^m \cdot \det A = \prod\limits_{k=1}^m \det A^k \),

                    где
                    \(A =\begin{pmatrix}
                        A^1    & 0      & \ldots & 0       \\
                        *      & A^2    & \ldots & 0       \\
                        \vdots & \vdots & \ddots & \vdots  \\
                        *      & *      & \ldots & A^{m-1}
                    \end{pmatrix} \)
          \end{itemize}

          Следствия:
          \begin{enumerate}
              \item \(\det\begin{pmatrix}
                        a_{11} & 0      \\
                        *      & a_{nn} \\
                    \end{pmatrix} = a_{11} \cdot \ldots \cdot a_{nn}\)\\
              \item  \(\rg{A} = n \Rightarrow \det A \neq 0\)

                    Просто преобразуем \(A\) методом Гаусса и получим трапециевидную. \(\rg{A} = n \Rightarrow\) после преобразований она будет треугольной, значит на диагонали нет нулей, значит их произведение не 0.
          \end{enumerate}

          Замечание: в силу свойства 1, всё сказанное верно и для верхнетреугольных матриц.

    \item \(\det A = \sum\limits_{i=1}^n a_{ij} A_{ij} = \), для какого-то столбца j.

          \( A_{ij} = ({-1})^{i+j} \cdot M_{ij}\), где \(M_{ij}\) - минор.

          \(A=\begin{pmatrix}
              I \ldots      & a_{1j} & II             \\
              a_{1n} \ldots & a_{ij} & \ldots  a_{in} \\
              III           & a_{mj} & IV
          \end{pmatrix}\), тогда \(M_{ij} = \det
          \begin{pmatrix}[c|c]
              I   & II \\
              \hline
              III & IV \\
          \end{pmatrix}\)

          Докажем сначала для 1 столбца:

          \(\det A = \sum\limits_{i = 1}^n (-1)^{i+1} A_{i1}\)

          \(\det A =
          \begin{vmatrix}
              a_{11} & *      & \ldots & *      \\
              a_{12} & *      & \ldots & *      \\
              \vdots & \vdots & \ddots & \vdots \\
              a_{1n} & *      & \ldots & *      \\
          \end{vmatrix} =
          \begin{vmatrix}
              a_{11} & *      & \ldots & *      \\
              0      & *      & \ldots & *      \\
              \vdots & \vdots & \ddots & \vdots \\
              0      & *      & \ldots & *      \\
          \end{vmatrix} +
          \begin{vmatrix}
              0      & *      & \ldots & *      \\
              a_{21} & a_{22} & \ldots & a_{2n} \\
              \vdots & \vdots & \ddots & \vdots \\
              0      & *      & \ldots & *      \\
          \end{vmatrix} + \cdots +
          \begin{vmatrix}
              0      & *      & \ldots & *      \\
              0      & *      & \ldots & *      \\
              \vdots & \vdots & \ddots & \vdots \\
              a_{n1} & a_{n2} & \ldots & a_{nn} \\
          \end{vmatrix} = a_{11} M_{11} - a_{21} M_{21} + a_{31} M_{31} - \ldots + (-1)^{n-1} a_{n1} M_{n1} = \sum\limits_{i = 1}^n (-1)^{i + 1} M_{i1} a_{i1}\)

          % здесь мы пользуемся свойством линейности и симметричности, раскрываем одну матрица на много матриц: TODO: подробнее расписать а то не понятно

          Докажем для произвольного j-ого столбца

          \(\det A = \det (\ldots A_j \ldots) = (-1)^{j-1} \det (A_j A_1 \ldots A_n) = \sum\limits_{i=1}^n(-1)^{j-1}(-1)^{i+1} a_{ij} M_{ij}\)




    \item \(\sum\limits_{i = 1}^n a_{ij} A_{ik} \) (\(j\) --- фиксированный номер столбца, \(k\) --- фиксированный номер другого столбца.) \( = 0 = \sum\limits_{j = 1}^n a_{ij} A_{kj}\) (\(i\) --- фиксированный номер строки, \(k\) --- фиксированный номер другой строки)


          Доказательство:

          \( \sum\limits_{i=1}^n a_{ij}A_{ik} = \det (A_1\ldots A_i \ldots A_j\ldots A_n)\)

          \(\det A = \sum\limits_{i=1}^n a_{ik}A_{ik} = \det (A_1\ldots A_k \ldots A_j\ldots A_n)\)


    \item \(\det(A \cdot B) = \det A \cdot \det B\)

          \(AB = (AB_1, \ldots, AB_n)\), \(B = (B_1, \ldots, B_n)\)

          \(\det(A \cdot B) = f(B_1, \ldots, B_n)\) (полилинейная, антисимметричная \(n\) - форма, \(f = \alpha D\))
          \(= f(E_1, \ldots, E_n) \cdot \det B = \det(A \cdot E) \cdot \det B = \det a \cdot \det B\)

\end{enumerate}
\subsection{Формула для обратной матрицы. Теорема Крамера.}

Матрица \(A_{n \times n}\) --- \textbf{невырожденная}, если \(\det A \neq 0\)

\textbf{Теорема:} (об обратной матрице) %TODO СТИЛЬ

Дано \(A_{n\times n}\). А обратима $\Leftrightarrow$ A невырождена. 

Причем, \(A^{-1}=\cfrac{1}{\det A} 
\begin{pmatrix}
    A_{11} & \ldots & A_{1n} \\
    \vdots & \ddots & \vdots \\
    A_{n1} & \ldots & A_{nn}
\end{pmatrix}^T\), \(A_{ij}\) --- алгебраическое дополнение элемента \(a_{ij}\)


Матрица в формуле называется союзной, взаимной, или присоединяемой.

Доказательство:
\begin{itemize}
\item \(\Rightarrow\) 

A обратима \(\Rightarrow\) \(\exists  A^{-1}\). \( A \cdot    A^{-1}  =  A^{-1} \cdot A = E \) 

\(\Rightarrow \det( A^{-1} A) =\det E = \det A^{-1} \cdot \det A \) m откуда уже следует искомое.

\item \(\Leftarrow\)

A - невырожденная. \(\det A \neq 0 \). Покажем, что матрица  \(B=\cfrac{1}{\det A} (A_{ij})^T\)



\(A \cdot B = \begin{pmatrix}
    a_{11} & \ldots & a_{1n} \\
    a_{21} & \ldots & a_{2n} \\
    \vdots & \ddots & \vdots \\
    a_{n1} & \ldots & a_{nn} \\
\end{pmatrix} \cdot 
\dfrac{1}{\det A} \cdot \begin{pmatrix}
    A_{11} & \ldots & A_{n1} \\
    A_{12} & \ldots & A_{n2} \\
    \vdots & \ddots & \vdots \\
    A_{1n} & \ldots & A_{nn} \\
\end{pmatrix} = \)

\(= \dfrac{1}{\det A} \cdot 
\begin{pmatrix}
    \sum\limits_{j = 1}^{n} a_{1j} A_{1j} & \ldots & \sum\limits_{j = 1}^{n} a_{1j} A_{nj} \\
    \sum\limits_{j = 1}^{n} a_{2j} A_{1j} & \ldots & \sum\limits_{j = 1}^{n} a_{2j} A_{nj} \\
    \vdots & \ddots & \vdots \\
    \sum\limits_{j = 1}^{n} a_{nj} A_{nj} & \ldots & \sum\limits_{j = 1}^{n} a_{nj} A_{nj}
\end{pmatrix} =\) 

(Все не диагональные ячейки по 8 свойству --- нули, а все диагональные по 7 свойству --- \(\det A\)) \(= E \Rightarrow B = A^{-1}\)
\end{itemize}

Следствия:
\begin{enumerate}
    \item \(A\) обратима \(\Leftrightarrow \rg A = n \Leftrightarrow det A \neq 0\)
    
    \item \(det A^{-1} = \cfrac{1}{\det A }\)
    
    \item Теорема Краммера
\end{enumerate}


\(Ax = b\). СЛНУ, \(A_{n\times n}\)

\(\exists! \) решение \(\Leftrightarrow A\) невырожденная.

Причём, \(x_i = \dfrac{\Delta_i}{\Delta}\), где \(\Delta = \det A\), \(\Delta_i = \det (A_1, \ldots, b, \ldots, A_n)\) (\(b\) занимает \(i\)-й столбец)


Доказательство:

\(\exists! \) решение \(\Leftrightarrow \rg A = n \Leftrightarrow \det A \neq 0\), то есть A - невырожденная

\(x = \cfrac{1}{\det A} \begin{pmatrix}
    A_{11} & \ldots & A_{n1} \\
    A_{12} & \ldots & A_{n2} \\
    \vdots & \ddots & \vdots \\
    A_{1n} & \ldots & A_{nn} \\
\end{pmatrix} \cdot 
 \begin{pmatrix}
   b_1 \\
   b_2 \\ 
   \vdots \\
   b_n
\end{pmatrix}
 = 
 \cfrac{1}{\det A}
 \begin{pmatrix}
   \sum\limits_{i=1}^n A_{i1}b_1 \\
  \sum\limits_{i=1}^n A_{i2}b_2  \\ 
   \vdots \\
   \sum\limits_{i=1}^n A_{i1}b_n 
\end{pmatrix}
 = 
 \cfrac{1}{\det A}
 \begin{pmatrix}
  \det (b,A_2\ldots A_n) \\
  \det (A_1,b\ldots A_n)  \\ 
   \vdots \\
   \det (A_1,A_2\ldots b)
\end{pmatrix} = 
 \begin{pmatrix}
  \cfrac{\Delta_1}{\Delta}\\
  \cfrac{\Delta_2}{\Delta}  \\ 
   \vdots \\
   \cfrac{\Delta_n}{\Delta}
\end{pmatrix} = x
\)
% возможно стоит переписать срочки или сделать какой-либо рефактор

\subsection{Теорема Лапласа}

\(A =(a_{ij})_{n \times n}\)

\(1 \leq k \leq n\): \(i_1 < i_2 < \ldots < i_k\), \(j_1 < j_2 < \ldots < j_k\)

\(i_s \in (1, \ldots, n)\), \(j_t \in (1, \ldots, n)\)

Составим из элементов матрицы A новую матрицу, состоящую из элементов, находящихся на пересечении k выбранных строк и k выбранных столбцов

Минор \(k\)-того порядка \(M_{i_1, \ldots, i_k}^{j_1, \ldots, j_k} =
\begin{vmatrix}
    a_{i_1 j_1} & \ldots & a_{i_n j_1} \\
    \vdots & \ddots & \vdots \\
    a_{i_1 j_n} & \ldots & a_{i_n j_n} \\
\end{vmatrix}\)

\(\overline{M}_{i_1, \ldots, i_k}^{j_1, \ldots, j_k} = M_{s_1, \ldots, s_m}^{t_1, \ldots, t_m}\)

\textbf{Теорема Лапласа.}

\(A_{n\times n }\), k - фиксированное от 1 до n. 

\(\det A = \sum\limits_{j_1<\ldots<j_k} \overline{M}_{j_1\ldots j_k}^{i_1\ldots i_k} A_{j_1\ldots j_k}^{i_1\ldots i_k}\)

Доказательство:

Пускай k выбрано от 1 до n и фиксирован набор строк. Тогда:

\(\sum\limits_{j_1<\ldots<j_k}(-1)^{i_1+i_2+\ldots+i_k+ j_1 +j_2 + \ldots j_k} \overline{M}_{j_1\ldots j_k}^{i_1\ldots i_k} M_{j_1\ldots j_k}^{i_1\ldots i_k} = \det A\)

%todo: мы вроде не пофиксили свойства определителя
\begin{itemize}
\item База индукции 

Свойство 7: \(\sum\limits_{j}(-1)^{i+j}\overline{M}_j^i M_j^i = \det A\)

\item Индукционное предположение

Пусть формула верна для \(k-1\):

Фиксированные \(i_1,\ldots,i_{k-1}: \det A = \sum\limits_{j_1<j_2<\ldots<j_{k-1}} (-1)^{i_1 + \ldots + i_{k-1}+j_1 + \ldots + j_{k-1}} \overline{M}_{j_1,\ldots, j_{k-1}}^{i_1,\ldots,i_{k-1}}M_{j_1,\ldots, j_{k-1}}^{i_1,\ldots,i_{k-1}}
\)


\item Индукционный переход:

Докажем для \(k\).

Пусть фиксированы \(i_1 < i_2 < \ldots < i_{k - 1} < i_k\). Всё кроме \(i_k\) верно по инд. предположению.

\(\overline{M}_{j_1, \ldots, j_{k - 1}}^{i_1, \ldots, i_{k - 1}} = M_{\ldots}^{\ldots, i_k, \ldots} = \sum\limits_{j \in (1, \ldots, n) \setminus (j_1, \ldots, j_{k-1})} a_{i_k j} (-1)^{\# i_k + \# i_j} \overline{M}_{j_1, \ldots, j_{k - 1}}^{i_1, \ldots, i_{k - 1}, i_k}\)

\(\# i_k = i_k - (k - 1)\)

\(\det A =\)

\(= \sum\limits_{j_1<\ldots<j_k} (-1)^{i_1+ \ldots+i_k-1 + j_1 + \ldots + j_k-1}\sum\limits_{j \in (1,\ldots,n)\setminus(j_1,\ldots,j_km)} (-1)^{i_k -(k-1) +\# j} a_{i_kj}\overline{M}_{j_1,j_2,\ldots,j_{k-1}}^{i_1 + \ldots i_k} M_{j_1,\ldots,j_{k-1}}^{i_1,\ldots i_{k-1}} \)

\(= \sum\limits_{\tilde j_1 < \ldots < \tilde j_k} (-1)^{i_1 + \ldots + i_k - (k - 1)} \sum\limits_{s = 1}^{k} a_{i_k \tilde j_s} \overline{M}_{\tilde j_1, \ldots, \tilde j_s, \ldots, \tilde j_k}^{i_1 \ldots i_{k - 1}, i_k} M_{\tilde j_1, \ldots, \tilde j_k}^{i_1 \ldots i_{k - 1}} (-1)^{\tilde j_s - (s - 1)} =\)

\(= \sum\limits_{\tilde j_1 < \ldots < \tilde j_k} (-1)^{i_1 + \ldots + i_k + \tilde j_1 + \ldots + \tilde j_k} \overline{M}_{\tilde j_1, \ldots, \tilde j_k}^{i_1, \ldots, i_k} \sum\limits_{s = 1}^{k} (-1)^{-(k - 1) - (s - 1)} a_{i_k \tilde j_s} M_{\tilde j_1, \ldots, \tilde j_k}^{i_1, \ldots, i_{k - 1}} =\)

\(= \sum\limits_{\tilde j_1 < \ldots < \tilde j_k} (-1)^{i_1 + \ldots + i_k + \tilde j_1 + \ldots + \tilde j_k} \overline{M}_{\tilde j_1, \ldots, \tilde j_k}^{i_1, \ldots, i_k} M_{\tilde j_1, \ldots, \tilde j_k}^{i_1, \ldots, i_k} = \det A\) --- верно и для \(k\)
\end{itemize}

\textbf{Замечание.}

\(\det \begin{pmatrix}[c|c]
A_1 & 0 \\
\hline 
* & A_2
\end{pmatrix} = \det A_1 \det A_2 = M_{j_1, \ldots, j_k}^{i_1, \ldots, i_k} \overline{M}_{j_1, \ldots, j_k}^{i_1, \ldots, i_k}\)


\subsection{Второе определение ранга матрицы.}

\(A_{m \times n}\)

\(M_{j_1\ldots j_k}^{i_1 \ldots i_k}\), 

\(\rg A\) называется наибольший порядок минора отличного от нуля, то есть \(\rg A = k\), если существует минор не равный нулю, а любой минор большего порядка равен 0. Такой минор является \textbf{базисным}, а строки и столбцы, входящие в этот минор --- \textbf{базисными}.

Базисный минор не определён единственным образом.   

\textbf{Замечание.}  Если все миноры \(k+1\) порядка 0, то все его миноры порядка больше \(k+1\) тоже 0.(очевидно из разложения по строчке или столбцу)

\textbf{Теорема} (об эквивалентности двух определений ранга)

Хотим доказать, что наши определения равносильны:

%из левого  в правую

\( \rg A_{\text{def 1}} = \rg A_{\text{def 2}}\)

\(\rg A_{\text{def 1}} = k\), \(1 \leq k \leq \min(m, n)\)

k линейно нез., k линейно нез.

Добавляя любые другие столбики к ним мы будем получать линейно зависимые комбинации.

Если отрезки зависимы и исходные стобцы зависимы.

Поэтому минор \(M_{j_1\ldots j_k}^{i_1\ldots i_k} = \det B\) - получается \(\rg B = k\), откуда получаем что определитель не 0.

Составим любой минор порядка \(k + s\).

\(M_{\tilde j_1, \ldots, \tilde j_{k + s}}^{\tilde i_1, \ldots, \tilde i_{k + s}} = \det(\tilde A_{\tilde j_1} \ldots \tilde A_{\tilde j_{k + s}})\) --- отрезки столбцов \(A\).

\(\rg A = k \Rightarrow \tilde A_{\tilde j_1} \ldots \tilde A_{\tilde j_{k + s}}\) --- линейно зависимы \(\Rightarrow\)

\(\Rightarrow  \det(\tilde A_{\tilde j_1} \ldots \tilde A_{\tilde j_{k + s}}) = M_{\tilde j_1, \ldots, \tilde j_{k + s}}^{\tilde i_1, \ldots, \tilde i_{k + s}} = 0 \Rightarrow \rg A_{\text{def 2}} = k\)

\uline{\textbf{Метод окаймляющих миноров.}}

\(A \neq \mathbb{0}\)

Алгоритм:

Берем смотрим на минор k-ого порядка:
\begin{enumerate}
    \item Если все его (окаймляющие прошлого этапа) миноры 0, то \(\rg A = k\).
    \item Если существует минор не равный 0, тогда k++ и повторить алгоритм 
\end{enumerate}

Окаймляющие миноры - миноры, в разложениях по строкам и столбцов которых присутствует данный минор

Пусть \(M_{j_1, \ldots, j_k}^{i_1, \ldots, i_k} \neq 0\), а все его окаймляющие его равны 0 \(\rg A = k\)

\(\forall i \forall j \notin(j_1, 
\ldots, j_k): 
\begin{vmatrix}
a_{i_1 j_1} & \ldots & a_{i_1 j_k} & a_{i_1 j} \\
a_{i_2 j_1} & \ldots & a_{i_2 j_k} & a_{i_2 j} \\
\vdots & \ddots & \vdots & \vdots \\
a_{i_k j_1} & \ldots & a_{i_k j_k} & a_{i_k j} \\
a_{i j_1} & \ldots & a_{i j_k} & a_{i j} \\
\end{vmatrix} = 0\) 

Если \(i\) совпадает с каким-либо индексом из \(i_1, \ldots, i_k\), то это определитель с равными строками, значит нулевой. Если \(i\) не совпадает ни с одним индексом из \(i_1, \ldots, i_k\), то тогда это окаймляющий минор \((k + 1)\)-го порядка, который нулевой по условию.

Распишем определитель по последней строке.

\( 0  = \sum\limits_{s=1}^{k} a_{i j_s} A_{ij_s} + a_{ij} (-1)^{k+1+k+1} M_{j_1\ldots j_k}^{i_1 \ldots i_k}\)

\(\forall i : a_{ij} =  0 \sum\limits_{s=1}^k a_{i j_s} A_{i,j_s} = \sum\limits_{s=1}^{k}a_{ij_S} \lambda_s \Leftrightarrow A_j = \sum\limits_{s=1}^{k}  A_{j_s} \lambda s\)

мы показали, что для \(\forall j \in \{j_1,\ldots,j_k\}\) --- линейная комбинация  соответствующих столбцов. 


\subsection{Определитель \(n\)-ого порядка.}


\textbf{Приведение к треугольному виду.}

\( \Delta_n \begin{pmatrix}
a_1 & x & x & \ldots & x \\
x & a_2 & x & \ldots & x \\
x & x & a_3 & \ldots & x \\
\vdots & \vdots & \vdots & \ddots & \vdots \\
x & x & x & \ldots & a_n \\
\end{pmatrix}
=
\begin{pmatrix}
a_1 & x & x & \ldots & x \\
x-a_1 & a_2 - x & 0 & \ldots & 0 \\
x-a_1 & 0 & a_3 -x & \ldots & 0 \\
\vdots & \vdots & \vdots & \ddots & \vdots \\
x-a_1 & 0& 0 & \ldots & a_n-x \\
\end{pmatrix}
=\)


\( = 
\prod\limits_{k=1}^n (a_k-x)
\begin{vmatrix}
\cfrac{a_1}{a_1-x} & \cfrac{x}{a_2-x} & \cfrac{x}{a_3-x} & \ldots & \cfrac{x}{a_n-x} \\
-1 & 1 & 0 & \ldots & 0 \\
-1 & 0 & 1 & \ldots & 0 \\
\vdots & \vdots & \vdots & \ddots & \vdots \\
-1 & 0& 0 & \ldots & 1 \\
\end{vmatrix}
 = \)
 
 \(=
\prod\limits_{k=1}^n (a_k-x)
\begin{vmatrix}
\sum & \cfrac{x}{a_2-x} & \cfrac{x}{a_3-x} & \ldots & \cfrac{x}{a_n-x} \\
0 & 1 & 0 & \ldots & 0 \\
0 & 0 & 1 & \ldots & 0 \\
\vdots & \vdots & \vdots & \ddots & \vdots \\
0 & 0& 0 & \ldots & 1 \\
\end{vmatrix}
\)
 Откуда уже можно легко посчитать определитель.

 \textbf{Метод выделения линейных множителей}

 
 \(
 \Delta_n = \begin{vmatrix}
    1 & x_1 & x_1^2 & \ldots & x_1^n \\
     1 & x_2 & x_2^2 & \ldots & x_2^n \\
      1 & x_3 & x_3^2 & \ldots & x_3^n \\
     \vdots & \vdots & \vdots & \ddots & \vdots \\
     1 & x_n & x_n^2 & \ldots & x_n^n \\
 \end{vmatrix} = p(x_i)
 \)
 Заметим, что когда \(x_i  = x_j \) определитель равен 0. Тогда получаем, что определитель должен делиться на каждый из корней( раскладывается в произведение корней)

  \(
 \Delta_n = \begin{vmatrix}
    1 & x_1 & x_1^2 & \ldots & x_1^n \\
     1 & x_2 & x_2^2 & \ldots & x_2^n \\
      1 & x_3 & x_3^2 & \ldots & x_3^n \\
     \vdots & \vdots & \vdots & \ddots & \vdots \\
     1 & x_n & x_n^2 & \ldots & x_n^n \\
 \end{vmatrix} = p(x_i) = (x_1 - x_2) \cdot (x_1  - x_3) \cdot \ldots \cdot (x_1 - x_n) \cdot (x_2 - x_3) \cdot \ldots \cdot (x_{n-1}-x_n) C 
 \)

\(\Delta_n = (x_n - x_1)(x_n - x_2)\ldots(x_n - x_{n - 1}) c' = \Delta_{n - 1} x_n^{n - 1} + \ldots\)

\(c' = \Delta_{n - 1}\)

\(\Delta_n = (x_n - x_1)(x_n - x_2)\ldots(x_n - x_{n - 1})(x_{n - 1} - x_n)\ldots(x_{n - 1} - x_{n - 2}) \Delta_{n - 2} =\)

\(= \prod\limits_{i > j}(x_i - x_j)\)

\textbf{Метод реккурентных соотношений}

 Возвратная последовательность. Пример.  \( x_2 =2 , x_1 = 4\). Реккурентная послеждовательность задается выражением \( x_n = x_{n-1} +2 x_{n-2}\) . И ее решая можно получить  корень 

Пример решения. \(x_1 =3, x_2 = 9, x_n = 3x_{n-1}-\cfrac{9}{4}x_{n-2}, n >2\).

Подставим вместо \(x_n = \lambda^n \) (не спрашивайте почему, там огромный кусок теорий и объяснений)

\( \lambda^n  = 3\lambda^{n-1} +  \cfrac{9}{4}\lambda^{n-2}\). Переведем в квадратное, решим, найдем корни. Получим \(\lambda_{1,2} = \cfrac{3}{2}\)

Тк лямбды совпали, то второй корень умножаем на n: %todo сделать пример 

\(x_n  = c_1 \Big(\cfrac{3}{2}\Big)^n  + c_2 n \Big (\cfrac{3}{2}\Big )^n \)

\(x_1  =c_1 (\cfrac{3}{2}) + c_2 \cfrac{3}{2};\) 
\(x_2  =c_1 (\cfrac{3}{2})^2 + c_2 2 (\cfrac{3}{2})^2\) 

Находим \(c_1,c_2\) и получаем общую реккуренту и побеждаем.

%TODO: ДОБАВИТЬ ЕЩЕ ОДИН ПРИМЕР









\(\Delta_n = \begin{vmatrix}
5 & 3 & 0 & \ldots & 0 \\
2 & 5 & 3 & \ldots & 0 \\
\vdots & \vdots & \vdots & \ddots & \vdots\\
0 & \ldots & 2 & 5 & 3\\
0 & \ldots & 0 & 2 & 5\\
\end{vmatrix} 
= 2 (-1)^{n + (n - 1)} \begin{vmatrix}
5 & 3 & 0 & \ldots & 0 \\
2 & 5 & 3 & \ldots & 0 \\
\vdots & \vdots & \vdots & \ddots & \vdots\\
0 & \ldots & 2 & 5 & 0\\
0 & \ldots & 0 & 2 & 3\\
\end{vmatrix} + 5(-1)^{n + n} \Delta{n - 1} = 
-6 \Delta_{n - 2} + 5\Delta_{n - 1}\)

\(\lambda^n + 6 \lambda^{n - 2} - 5\lambda^{n - 1} = 0\)

\(\lambda^2 - 5 \lambda + 6 = 0\)

\(\lambda_1 = 2\), \(\lambda_2 = 3\)

\(\Delta_n =2^nc_1 + 3^nc_2\)

\(
\begin{cases}

    \Delta_1  =5  =2 c_1 + 3 c_2 \\
     
    \Delta_2 = \begin{vmatrix}
        5 & 3 \\
        2 & 5 \\
    \end{vmatrix}
    = 19 = 4c_1  + 9c_@
\end{cases}\)


Откуда \(\Delta_n = -2^{n+1} + 3^{n+1}\).

\textbf{Метод представления в виде суммы}

\( \Delta_n = \begin{vmatrix}
    a_1 + b_1 & a_1+b_2 & \ldots & a_1 + b_n \\
    a_2 + b_1 & a_2+b_2 & \ldots & a_2 + b_n \\
    \vdots & \vdots & \ddots & \vdots \\
     a_n + b_1 & a_n+b_2 & \ldots & a_n + b_n \\
\end{vmatrix}\) Разложим определитель, как сумму определителей. 

%ну что поехали
\(
 \begin{vmatrix}
    a_1 + b_1 & a_1+b_2 & \ldots & a_1 + b_n \\
    a_2 + b_1 & a_2+b_2 & \ldots & a_2 + b_n \\
    \vdots & \vdots & \ddots & \vdots \\
     a_n + b_1 & a_n+b_2 & \ldots & a_n + b_n \\
\end{vmatrix} = \begin{vmatrix}
    a_1 & a_1 & \ldots & a_1  \\
    a_2 + b_1 & a_2+b_2 & \ldots & a_2 + b_n \\
    \vdots & \vdots & \ddots & \vdots \\
     a_n + b_1 & a_n+b_2 & \ldots & a_n + b_n \\
\end{vmatrix}  +  \begin{vmatrix}
     b_1 & b_2 & \ldots &  b_n \\
    a_2 + b_1 & a_2+b_2 & \ldots & a_2 + b_n \\
    \vdots & \vdots & \ddots & \vdots \\
     a_n + b_1 & a_n+b_2 & \ldots & a_n + b_n \\
\end{vmatrix}  \)
%TODO: Раздать стилька

Повторяя так большее количество так и зануляя определители победим.

\textbf{Метод  изменения элементов det}

\(
\Delta = \begin{vmatrix}
    a_{11} & \ldots & a_{1n} \\
    \vdots & \ddots & \vdots \\
     a_{n1} & \ldots & a_{nn} \\
\end{vmatrix}
\)

\(
\Delta' =  \begin{vmatrix}
    a_{11}+ x & \ldots & a_{1n} + x \\
    a_{21}+ x & \ldots & a_{2n} + x \\
    \vdots & \ddots & \vdots \\
    a_{n1}+ x & \ldots & a_{nn} + x \\
\end{vmatrix}
\) Начну раскладывать, как в прошлом пункте и получу, что 

\(
\Delta' = \Delta + x \sum\limits_{j,j=1}^n A_{ij}\)
%TOFO: ПРИМЕР МНЕ ВПАДЛУ ЧЕСТНО 


