\subsection{Аксиматическое вероятное пространство .}

Пусть у нас есть $\Omega$ - \deff{элементарные исходы} и связанная с ним функция \\ $p: \Omega \rightarrow [0,1]$  - \deff{дискретная вероятностная мера (плотность вероятности)} - функция, которая по элементарному исходу возвращает \deff{вероятность}.

А также $\sum\limits_{w \in \Omega} p(w) = 1$, а также $0 \leq p_i \leq 1$  А также мы считаем, что $|\Omega|$ не более чем счетно. Для множеств мощности континуума нам нужна более сложная теория.  

Рассмотрим \textbf{примеры}:

\begin{enumerate}
    \item {Честная монета:}

    $\Omega = \{0,1\}$. $p(0) = p(1) = \frac{1}{2}$.

    \item {Нечестная монета} или распределение Бернулли:
    
    $\Omega = \{0,1\}$. $p(0) = 1 - p(1) = q$.

     \item {Честная игральная кость:}

    $\Omega = \{1,2,3,4,5,6\}$. $p(w) = \frac{1}{6}$. $p(w) = \cfrac{1}{52}$

    \item Колода карт:

    $\Omega = \{\langle c, r\rangle\, 1\leq c \leq 4, 1\leq r \leq 15\}$

    \item Геометрическое распределение:

    $\Omega = \mathbb{N}$, $p(i) = \cfrac{1}{2^i}$
    
\end{enumerate}

\textbf{Замечание.} Не существует равномерного распределения на счетном множестве.

\deff{Событие} ---  множество $A \subset \Omega$. $P(A) = \sum\limits_{w \in A} p(w)$. (Иногда используют Pr).

$P(A) = 1 $ --- достоверное событие.

$P(A) = 0 $ --- невозможное событие.

Рассмотрим примеры на честной игральной кости:


\begin{enumerate}
\item Только четные: $P(A) = \frac{3}{6}=\frac{1}{2}$.
\item Больше 4-ех: $P(A) = \frac{2}{6}=\frac{1}{3}$.
\end{enumerate}

\textbf{Замечание:} нельзя с равной вероятностью выбрать случайное целое число. 

\deff{Независимые события} --- A,B независимы, если $P(A \cap B) = P(A) \cdot P(B)$ .

$\cfrac{P(A \cap B)}{ P(B)}  = \cfrac{P(A)}{P(\Omega)}$ --- независимы (если выполнилось B, то вероятность не поменялась)

$P(A | B) = \cfrac{P(A \cap B)}{P(B)}$ --- вероятность A при условии B ---
\deff{условная вероятность}.

\deff{Произведение вероятностных пространств.} 

Пусть у нас есть $\Omega_1. p_1$, а также $\Omega_2,p_2$, тогда произведение вероятностных пространств:

\[\Omega = \Omega_1 \times \Omega_2\] \[p(\langle w1,w2\rangle) = p_1(w_1) \cdot p_2(w_2)\].

\textbf{Утв.} $\forall A \subset \Omega_1, B \subset \Omega_2$.

$A \times \Omega_2 $ и $\Omega_1 \times B $  независимы.

Пусть у нас есть n - событий: $A_1, A_2, \ldots, A_n$.

Тогда обычно \deff{независимость n событий} подразумевает:

\begin{enumerate}
    \item $A_i, A_j $ - независимы $\forall i,j$
    \item $\forall I \subset \{1,2,3,\ldots, n\}$. $P(\bigcup\limits_{i \in I }A_i) = \prod\limits_{i \in I} P(A_i)$
\end{enumerate}

\deff{Формула полной вероятности}

$\Omega = A_1 \cup A_2 \cup \ldots \cup A_n $, $\forall i \neq j: A_i \cap A_j = \o$ --- \deff{полная система событий}.

Возьму B - какое-то событие.

$P(B) = \sum\limits_{i=1}^n P(B \cap A_i) =  \sum\limits_{i=1}^n P(B| A_i) \cdot P(A_i)$

Пример: Урна с шариками. Сначала выбираете урну, потом достаете шарик.

\deff{Формула Байеса.}

$P(A_i | B) = \cfrac{P(A_i\cap B)}{P(B)} = \cfrac{P(B| A_i) \cdot P(A_i)}{\sum\limits_{j=1}^n P(B| A_j) \cdot P(A_j)}$

