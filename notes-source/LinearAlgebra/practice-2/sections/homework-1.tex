\subsubsection{Задача 1.}

На пространстве $P_2$ --- многочленов степени не выше второй заданы две системы линейных форм $f^i$ и $g^j$:
$$\forall p \in P_2: f^i(p) = p(i), i=1,2,3$$
$$\forall p \in P_2: g^j = p^{(j-1)}(2), j=1,2,3$$
\begin{enumerate}
    \item Проверить, что каждая из систем является базисом в пространстве $(P_2)^*$
    \item Построить сопряженные базисы к каждой из систем
    \item Найти матрицы $S$ и $T$
    \item Написать ковариантный и контрвариантный законы преобразования координат
\end{enumerate}

\textbf{Решение:}

Сперва распишем все в сопряженным к базису $e_1,e_2,e_3$. Для этого подставим в $f^i,g^j$ $e_1,e_2,e_3$.

$a_{f^1} =(1,1,1), a_{f^2} =(1,2,4), a_{f^3}=(1,3,9)$

$a_{g^1 }= (1,2,4), a_{g^2}=(0,1,4), a_{g^3}= (0,0, 2)$

В принципе видно, что ранги системы векторов равны 3 в обоих случаях, откуда базисы в $(P_2)^*$

Теперь напишем $S_f  = \begin{pmatrix}
    1 & 1 & 1 \\
    1 & 2 & 4 \\
    1 & 3 & 9 
\end{pmatrix}, S_g = \begin{pmatrix}
    1 & 2 & 4 \\
    0 & 1 & 4 \\
    0 & 0 & 2
\end{pmatrix}, T_f = S_f^{-1}=\begin{pmatrix}
    3 & -3 & 1 \\
    -2,5 & 4 & -1,5\\
    0,5 & -1 & 0,5
\end{pmatrix}, T_g = S_g^{-1}=\begin{pmatrix}
    1 & -2 & 2\\
    0 & 1 & -2 \\
    0 & 0 &  0,5
\end{pmatrix}$

Теперь напишем ковариантный и контрвариантный законы координат:

$x' = Sx$ - контрвариатное.
$a' = aT$ - ковариантное.