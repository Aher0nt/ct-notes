
\subsection{Нижняя граница на размер функциональной схемы}
\textbf{Линейные программы} - еще один способ задать булеву функцию, в котором удобнее ее рассматривать для всяких оценок. Выберем какой-то базис и переменный, $n$ из которых будут начальными, которые мы будем получать на вход, для остальных заведем правила, по которым они будут вычисляться из предыдущих (расположены эти элементы будут в топологическом порядке).

Пусть у нас есть всего 1 операция в базисе - стрелка Пирса, тогда давайте оценим количество возможных линейных программ для $n$-арной функции с $t$ операциями. Первая операция может быть одной из $n^2$, вторая одной из $(n+1)^2$ и так далее. Тогда сумма: $$n^2*(n+1)^2\dots *(n+t)^2\leq(n+t)^{2t}$$

Подставим в эту формулу константу, например $\frac{2^n}{3n}$: $$\displaystyle\left(n+\frac{2^n}{3n}\right)^{\frac{2*2^n}{3n}}$$

Теперь если мы подсчитаем отношения этого количества ко всем возможным функциям, а их я напомню - $2^{2^n}$, то поймете, что оно довольно быстро стремится к 0.

\subsection{Верхняя граница на размер функциональной схемы}
Пусть у нас есть $n$-арная булева функция для которой мы хотим построить функциональную схему. Заведем две переменные $k$ и $s$ от которых в итоге и будет зависить размер нашей схемы. 

Начнем с того, что представим нашу таблицу истинности, как таблицу c $2^k$ строчками и $2^{n-k}$ столбцами, просто выделив $n-k$ элементов как столбцы. Теперь объединим строчки в блоки по $s$ строк. Хорошо, давайте научимся строить схему для таблицы из одного столбца и $s$ строк. Вариантов, в каком блоке такая подтаблица может оказаться - $\frac{2^k}{s}$, вариантов для каждого блока - $2^s$. Построим демультиплексор, на $k$ элементах, чтобы определять в какой из строк мы оказались, на это уйдет $2^k$ элементов схемы, далее для каждого из типов блоков соберем элементы или, которые в сумме позволят его получить, на это уйдет максимум $s$  умноженное на количество вариаций блоков, то есть: $$\frac{2^k}{s}*2^s*s=2^{k+s}$$

Хорошо, теперь нам нужно собрать столбец целиком, для этого мы возьмем полученные блоки и сделаем с нужными или, всего элеметов или у нас на это уйдет: $$2^{n-k}*\frac{2^k}{s}$$
Осталось собрать столбики в таблицу, чтобы это сделать снова построим демультиплексор, но теперь на $n-k$ вершинах, на что уйдет $2^{n-k}$ элементов, а потом сделаем или для каждого выхода демультиплексора, с нужным для него столбцом, на что уйдет всего $2^{n-k}$ элементов. После этого все результаты надо объединить за еще $2^{n-k}$ элементов.

Получается всего мы потратили: $$2^k+2^{n-k}+2^{k+s}+2^{n-k}*\frac{2^k}{s}+2^{n-k}=O(2^{k+s}+\frac{2^n}{s})$$
Если выбрать $k=\log_2n$, а $k=n-2\log_2n$, то мы и получим сложность порядка $O(\frac{2^n}{n})$.
\