\deff{Поглощающий эргодичесткий класс} -- класс, из которог в конденсации нет рёбер



МЦ \deff{регулярная} если $\forall i, j \ P_{i,j} > 0$



\thmm{Теорема: Эргодическая теорема для марковских цепей}

    $\exists b \ \forall b^{(0)}P^{(n)} \xrightarrow[n\rightarrow\infty]{}b $ и $bP = b$
    
    \textbf{Доказательство:}
    
    Давайте возьмём наш вектор $b^{(0)}$ и умножим на матрицу $A$ такую, 
    что $A \in M_{n \times n}$ и $\forall j \ a_{ji} = \tilde{a}$ 
    
    $(b \cdot A)_i = \sum_{j = 1}^n b_j a_{ji} = (\sum_{j = 1}^n b^{(0)})\tilde{a_i} = \tilde{a_i}$.
    т.к. сумма в столбике $b^{(0)}$ всегда равна единичке

    Докажем, что $P^n\xrightarrow[n\rightarrow\infty]{} A$, удовлетворяющей условиям.
    
    Введем $M_i^{(t)} = max(P^t)_{ji}, \ m_i^{(t)} = min(P^t)_{ji}$, посмотрим на разность максимума и минимума в столбце, правда ли он стремится к нулю?

    Возьмём $\delta = \max P_{ij}$ и $\tilde{\delta} = \min P_{ij}$  
    \begin{multline*}
    (P^{(t+1)})_{ij} = (P \,\cdot\,P^t)_{ij} = \sum_{k = 1}^n P_{jk}(P^t)_{ki} \le \sum_{n != pos(min)}P_{jk}\,\cdot\,M_i^{(t)} + P_{j, pos(min)}m_i^{(t)} =\\ = \sum_{k=1}^nP_{ik}\,\cdot\,M_i^{(t)} + P_{j, pos(min)}(m_i^{(t)} - M_i^{(t)}) \le M_i^{(t)} -\delta A^{(t)}
    \end{multline*}
    с другой стороны
    \begin{multline*}
    (P^{(t+1)})_{ij} = (P \,\cdot\,P^t)_{ij} = \sum_{k = 1}^n P_{jk}(P^t)_{ki} \ge \sum_{k = 1}^n P_{jk}m_i^{(t)} +P_{j, pos(max)}(M_i^{(t)} - m_i^{(t)})= m_i^{(t)} + \tilde{\delta} A_i^{(t)}
    \end{multline*}

    $$M_i^{(t+1)} \le M_i^{(t)} - \delta A_i^{(t)}$$
    $$m_i^{(t+1)} \ge m_i^{(t)} + \tilde{\delta}A_i^{(t)}$$
    минусуем

    $$A_i^{(t+1)} \le A_i^{(t)} - (\delta - \tilde{\delta} A_i^{(t)})  = (1 - \delta - \tilde{\delta})A_i^{(t)} \le \ldots \le (1 - \delta - \tilde{\delta})^{(t+1)} \rightarrow \mathbf{0}$$

    Если разность стремится к 0, то в 
    $b^{(0)}P^{(n)} \xrightarrow[n\rightarrow\infty]{} b^0A = b$

\hfill Q.E.D.

Давайте научимся искать $b$, решим следующее уравнение: $(I - P)b = \mathbf{0}$. Казалось бы, это СЛОУ и она может иметь либо 0 решений, либо бесконечно много, тогда почему же у нас есть конкретное решение?
На самом деле, $(I-P)$ -- вырожденная, и она имеет ранг $n -1$ и у нас одномерное пространство решений, но нам подойдёт только тот вектор, у которого сумма координат единичка. Тогда как в итоге найти $b$?

У нас есть два способа, мы либо возведём много раз матрицу $P$ в квадрат, с какого-то момента это всё будет стремиться к A, мы найдём ответ и, тогда сможем вычислить $b$, все операции займут у нас $O(n^3c)$, понятно, что если использовать какие-то продвинутые алгоритмы, оценку можно улучшить.

Либо же просто решим систему уравнений, но это тоже имеет свои тонкости, как минимум нам понадобится работать с вещественными числами, а мы знаем, что вещественные числа в компьютере несовершенны и дорого с ними работать, хоть и асимтотика номинально $O(n^3)$, но она упирается в проблемы работы с даблами. 

Наши теоремы мы доказали только для регулярных МЦ, но что происходит с нерегулярными? 

\thmm{Теорема}

    Если P -- матрица нерегулярного эргодического класса, тогда $\exists t : P^t$ - регулярное
    Доказывать мы это не будет, просто поверим на слово 







