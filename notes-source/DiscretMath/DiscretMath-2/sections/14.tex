\subsection{Детерминированные автоматы с магазинной памятью.}

Автомат называется \deff{детерминированным}, если $\forall Q,c :|\delta(c,A)|\leq 1$. 

Если $|\delta(\varepsilon,A)|=1$, то $\forall c \in \Sigma: |\delta(q,c,A)| = 0$

\thmm{Теорема.}

$A$ язык $\exists$ детерминированный автомат с магазинной памятью с допуском по пустому стеку.

\begin{enumerate}
    \item существует ДМП-автомат с допуском по допускающим состояний.
    \item $\forall x \in A, y\in A, x \neq y,x$ на префиксе $y$.
\end{enumerate}


\subsection{Существенно неоднозначные языки.}

\deff{Существенно неодназначный} языком называется язык такой, что любая грамматика не может задаться неодназными языками

\thmm{Лемма (Огдена)}

Пусть $L$-КС язык, тогда $\exists k  >0: \forall w \in L: |w|  >n, 1\leq i_1<2<i_n\leq |w|$, то существует $w = uvxyz, u,v,x$ содержит пометки или $x,y,z$ содержит пометки.

$\forall k \geq 0: uv ^kxy^kz \in L$