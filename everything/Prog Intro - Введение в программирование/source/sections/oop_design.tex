\subsection{SOLID}

Принципы \texttt{SOLID} --- это набор из пяти основных принципов объектно-ориентированного программирования и проектирования, которые помогают разработчикам создавать более гибкие, расширяемые и поддерживаемые системы.


\subsubsection{SRP: Принцип единственной ответственности (Single Responsibility Principle)}

\textbf{Определение:}
Каждый класс должен иметь одну и только одну причину для изменения, то есть класс должен иметь только одну ответственность.

\textbf{Пояснение:}
Если класс выполняет несколько задач, изменения в одной части могут повлиять на другие. Это усложняет тестирование и поддержку кода. Принцип SRP призывает разделять функциональность по классам таким образом, чтобы каждый класс отвечал за свою конкретную часть логики.

\textbf{Пример:}

\textit{Нарушение SRP:}
\begin{minted}{java}
public class Employee {
    private String name;
    private double salary;

    public void calculateSalary() {
        // Логика расчета зарплаты
    }

    public void saveToDatabase() {
        // Логика сохранения в базу данных
    }
}
\end{minted}

В этом примере класс \texttt{Employee} отвечает и за бизнес-логику (расчет зарплаты), и за сохранение данных, что нарушает \texttt{SRP}.

\textit{Следование SRP:}
\begin{minted}{java}
public class Employee {
    private String name;
    private double salary;

    public void calculateSalary() {
        // Логика расчета зарплаты
    }
}

public class EmployeeRepository {
    public void save(Employee employee) {
        // Логика сохранения сотрудника в базу данных
    }
}
\end{minted}

Теперь каждый класс имеет свою единственную ответственность: \texttt{Employee} --- бизнес-логика сотрудника, \texttt{EmployeeRepository} --- сохранение данных.

\subsubsection{OCP: Принцип открытости/закрытости (Open/Closed Principle)}

\textbf{Определение:}
Программные сущности должны быть \textbf{открыты для расширения}, но \textbf{закрыты для модификации}.

\textbf{Пояснение:}
Новый функционал должен добавляться путем создания нового кода, а не изменения существующего. Это снижает риск внесения ошибок в уже проверенный код.

\textbf{Пример:}

\textit{Нарушение OCP:}
\begin{minted}{java}
public class ShapeDrawer {
    public void drawShape(Object shape) {
        if (shape instanceof Circle) {
            drawCircle((Circle) shape);
        } else if (shape instanceof Square) {
            drawSquare((Square) shape);
        }
        // При добавлении новой фигуры потребуется изменить этот метод
    }

    private void drawCircle(Circle circle) {
        // Логика рисования круга
    }

    private void drawSquare(Square square) {
        // Логика рисования квадрата
    }
}
\end{minted}

\textit{Следование OCP:}
\begin{minted}{java}
public interface Shape {
    void draw();
}

public class Circle implements Shape {
    @Override
    public void draw() {
        // Логика рисования круга
    }
}

public class Square implements Shape {
    @Override
    public void draw() {
        // Логика рисования квадрата
    }
}

public class ShapeDrawer {
    public void drawShape(Shape shape) {
        shape.draw();
    }
}
\end{minted}

Теперь для добавления новой фигуры достаточно создать новый класс, реализующий \texttt{Shape}, не изменяя существующий код.

\subsubsection{LSP: Принцип подстановки Лисков (Liskov Substitution Principle)}

\textbf{Определение:}
Объекты базового класса должны быть заменяемы объектами производных классов без нарушения работы программы.

\textbf{Пояснение:}
Это означает, что подкласс должен дополнять, а не изменять поведение базового класса. Нарушение LSP приводит к неожиданным ошибкам при использовании полиморфизма.

\textbf{Пример:}

\textit{Проблема с прямоугольником и квадратом:}
\begin{minted}{java}
public class Rectangle {
    protected int width;
    protected int height;

    public void setWidth(int width) {
        this.width = width;
    }

    public void setHeight(int height) {
        this.height = height;
    }

    public int getArea() {
        return width * height;
    }
}

public class Square extends Rectangle {
    @Override
    public void setWidth(int size) {
        this.width = size;
        this.height = size;
    }

    @Override
    public void setHeight(int size) {
        this.width = size;
        this.height = size;
    }
}
\end{minted}

Используя полиморфизм, ожидаем, что \texttt{Rectangle} и \texttt{Square} будут работать одинаково, но изменение ширины или высоты \texttt{Square} изменяет обе стороны, что может привести к непредвиденным последствиям.

\subsubsection{ISP: Принцип разделения интерфейса (Interface Segregation Principle)}

\textbf{Определение:}
Клиенты не должны зависеть от методов, которые они не используют.

\textbf{Пояснение:}
Большие интерфейсы следует разделять на более маленькие и специфичные, чтобы клиенты зависели только от тех методов, которые им действительно нужны.

\textbf{Пример:}

\textit{Нарушение ISP:}
\begin{minted}{java}
public interface Worker {
    void work();
    void eat();
}

public class Robot implements Worker {
    @Override
    public void work() {
        // Робот работает
    }

    @Override
    public void eat() {
        // Роботу не нужно есть, но он обязан реализовать этот метод
    }
}
\end{minted}

\textit{Следование ISP:}
\begin{minted}{java}
public interface Workable {
    void work();
}

public interface Eatable {
    void eat();
}

public class Human implements Workable, Eatable {
    @Override
    public void work() {
        // Человек работает
    }

    @Override
    public void eat() {
        // Человек ест
    }
}

public class Robot implements Workable {
    @Override
    public void work() {
        // Робот работает
    }
}
\end{minted}

Теперь \texttt{Robot} не обязан реализовывать метод \texttt{eat()}.

\subsubsection{DIP: Принцип инверсии зависимостей (Dependency Inversion Principle)}

\textbf{Определение:}
Модули верхнего уровня не должны зависеть от модулей нижнего уровня. Оба должны зависеть от абстракций.

\textbf{Пояснение:}
Код должен зависеть от абстракций (интерфейсов), а не от конкретных реализаций. Это повышает модульность и упрощает тестирование.

\textbf{Пример:}

\textit{Нарушение DIP:}
\begin{minted}{java}
public class MySQLDatabase {
    public void connect() {
        // Подключение к MySQL
    }
}

public class Application {
    private MySQLDatabase database;

    public Application() {
        database = new MySQLDatabase();
    }

    public void start() {
        database.connect();
    }
}
    \end{minted}
    
    \textit{Следование DIP:}
    \begin{minted}{java}
public interface Database {
    void connect();
}

public class MySQLDatabase implements Database {
    @Override
    public void connect() {
        // Подключение к MySQL
    }
}

public class PostgreSQLDatabase implements Database {
    @Override
    public void connect() {
        // Подключение к PostgreSQL
    }
}

public class Application {
    private Database database;

    public Application(Database database) {
        this.database = database;
    }

    public void start() {
        database.connect();
    }
}
\end{minted}
    
Теперь \texttt{Application} зависит от абстракции \texttt{Database}, и мы можем подставить любую реализацию.



\subsection{Квадрат и прямоугольник}


\subsubsection{Постановка задачи}

Рассмотреть, как правильно спроектировать отношения между классами \texttt{Rectangle} (прямоугольник) и \texttt{Square} (квадрат) в соответствии с принципами ООП, особенно с учетом принципа подстановки Лисков.

\subsubsection{Источник проблем}
\begin{enumerate}
    \item \textbf{Наследование:} С математической точки зрения, квадрат является частным случаем прямоугольника.
    \item \textbf{Принцип подстановки Лисков:} Подкласс должен полностью соответствовать поведению базового класса.
    \item \textbf{Изменение свойств:} У квадрата ширина и высота всегда равны, что нарушает логику изменения отдельных свойств \texttt{width} и \texttt{height} в \texttt{Rectangle}.
\end{enumerate}

\subsubsection{Возможные решения}
\begin{enumerate}
    \item \textbf{Отказ от изменений}
    \begin{enumerate}
        \item \textbf{Возврат нового значения}
        
        Вместо изменения состояния объекта, методы возвращают новый объект с измененным состоянием.
        \begin{minted}{java}
        public class ImmutableRectangle {
            private final int width;
            private final int height;

            public ImmutableRectangle(int width, int height) {
                this.width = width;
                this.height = height;
            }

            public ImmutableRectangle setWidth(int width) {
                return new ImmutableRectangle(width, this.height);
            }

            public ImmutableRectangle setHeight(int height) {
                return new ImmutableRectangle(this.width, height);
            }
        }
        \end{minted}
        
        \item \textbf{Возврат флага}
        
        Методы изменения возвращают булевый флаг, указывающий на успешность операции.
        \begin{minted}{java}
        public class Square extends Rectangle {
            @Override
            public boolean setWidth(int width) {
                if (width == this.height) {
                    this.width = width;
                    return true;
                }
                return false;
            }
        }
        \end{minted}
        
        \item \textbf{Исключения}
        
        Бросать исключение при попытке установить некорректное значение.
        \begin{minted}{java}
        public class Square extends Rectangle {
            @Override
            public void setWidth(int width) {
                if (width != this.height) {
                    throw new IllegalArgumentException("Width and height must be equal");
                }
                this.width = width;
            }
        }
        \end{minted}
        
    \end{enumerate}
    
    \item \textbf{Отказ от наследования}
    \begin{enumerate}
        \item \textbf{Полный отказ от наследования}
        
        \texttt{Square} и \texttt{Rectangle} не связаны отношением наследования.
        \begin{minted}{java}
        public class Rectangle {
            private int width;
            private int height;
            // Геттеры и сеттеры
        }

        public class Square {
            private int side;
            // Геттеры и сеттеры
        }
        \end{minted}
        
        \item \textbf{Выделение общего базового класса}
        
        Создать абстрактный класс или интерфейс \texttt{Quadrilateral} (четырехугольник).
        \begin{minted}{java}
        public abstract class Quadrilateral {
            public abstract int getArea();
        }

        public class Rectangle extends Quadrilateral {
            private int width;
            private int height;
            // Реализация методов
        }

        public class Square extends Quadrilateral {
            private int side;
            // Реализация методов
        }
        \end{minted}
        
    \end{enumerate}
    
    \item \textbf{Дополнительные действия}
    
    Изменение логики методов базового класса для учета особенностей подклассов.
    
    \item \textbf{Выделение модифицируемых сущностей}
    
    Разделить объекты на изменяемые и неизменяемые, применяя соответствующие подходы к каждому.
    
    \item \textbf{Отказ от квадратов}
    
    Не создавать отдельный класс \texttt{Square}, а управлять этим внутри класса \texttt{Rectangle}.
    \begin{minted}{java}
    public class Rectangle {
        private int width;
        private int height;

        public static Rectangle createSquare(int side) {
            return new Rectangle(side, side);
        }
    }
    \end{minted}
    
\end{enumerate}


\subsection{Равенство}

\subsubsection{Свойства равенства}

Метод \texttt{equals} должен удовлетворять следующим свойствам:
\begin{enumerate}
    \item \textbf{Рефлексивность:} Для любого ненулевого значения \texttt{x}, \texttt{x.equals(x)} должно возвращать \texttt{true}.
    \item \textbf{Симметричность:} Для любых ненулевых значений \texttt{x} и \texttt{y}, \texttt{x.equals(y)} должно возвращать \texttt{true} тогда и только тогда, когда \texttt{y.equals(x)} возвращает \texttt{true}.
    \item \textbf{Транзитивность:} Если \texttt{x.equals(y)} и \texttt{y.equals(z)} возвращают \texttt{true}, то \texttt{x.equals(z)} должно возвращать \texttt{true}.
    \item \textbf{Согласованность:} Многократные вызовы \texttt{x.equals(y)} должны возвращать одинаковый результат, если объекты не изменились.
    \item \textbf{Неравенство с \texttt{null}:} Для любого ненулевого значения \texttt{x}, \texttt{x.equals(null)} должно возвращать \texttt{false}.
\end{enumerate}

\subsubsection{Метод equals}

\textbf{Переопределение \texttt{equals}:}
\begin{minted}{java}
public class Person {
    private String name;
    private int age;

    // Конструктор, геттеры и сеттеры

    @Override
    public boolean equals(Object obj) {
        if (this == obj) return true;
        if (obj == null || getClass() != obj.getClass()) return false;
        Person person = (Person) obj;
        return age == person.age &&
               Objects.equals(name, person.name);
    }
}
\end{minted}

\textbf{Пояснение:}
\begin{enumerate}
    \item Проверяем, ссылаются ли объекты на одну область памяти.
    \item Проверяем, не является ли объект \texttt{null} и совпадают ли классы.
    \item Приводим объект и сравниваем поля.
\end{enumerate}

\subsubsection{Метод hashCode}

\textbf{Переопределение \texttt{hashCode}:}
\begin{minted}{java}
@Override
public int hashCode() {
    return Objects.hash(name, age);
}
\end{minted}

\textbf{Пояснение:}
\begin{enumerate}
    \item Метод \texttt{hashCode} должен возвращать одинаковое значение для объектов, которые равны по \texttt{equals}.
    \item Это важно для корректной работы хеш-таблиц, таких как \texttt{HashMap} и \texttt{HashSet}.
\end{enumerate}

\subsubsection{Взаимодействие с наследованием}
\begin{enumerate}
    \item \textbf{Наивная реализация}
    
    При переопределении \texttt{equals} в подклассе могут возникнуть проблемы с симметричностью и транзитивностью.
    
    \textbf{Пример проблемы:}
    \begin{minted}{java}
    public class Point {
        protected int x, y;

        // Переопределение equals и hashCode
    }

    public class ColorPoint extends Point {
        private String color;

        // Переопределение equals и hashCode
    }
    \end{minted}
    
    Если \texttt{ColorPoint} сравнивается с \texttt{Point}, могут возникнуть несоответствия.
    
    \item \textbf{Использование сравнения предка}
    
    При переопределении \texttt{equals} в подклассе следует использовать \texttt{super.equals(obj)} для сравнения полей суперкласса.
    
    \textbf{Пример:}
    \begin{minted}{java}
    @Override
    public boolean equals(Object obj) {
        if (!super.equals(obj)) return false;
        if (getClass() != obj.getClass()) return false;
        ColorPoint that = (ColorPoint) obj;
        return Objects.equals(color, that.color);
    }
    \end{minted}
    
    \item \textbf{Сегрегация сравнения}
    
    Использование метода \texttt{canEqual} для проверки совместимости классов.
    
    \textbf{Пример:}
    \begin{minted}{java}
    public class Point {
        // Поля, конструкторы

        @Override
        public boolean equals(Object obj) {
            if (this == obj) return true;
            if (!(obj instanceof Point)) return false;
            Point point = (Point) obj;
            return point.canEqual(this) && x == point.x && y == point.y;
        }

        public boolean canEqual(Object obj) {
            return obj instanceof Point;
        }
    }

    public class ColorPoint extends Point {
        private String color;

        // Конструкторы

        @Override
        public boolean equals(Object obj) {
            if (this == obj) return true;
            if (!(obj instanceof ColorPoint)) return false;
            if (!super.equals(obj)) return false;
            ColorPoint that = (ColorPoint) obj;
            return that.canEqual(this) && Objects.equals(color, that.color);
        }

        @Override
        public boolean canEqual(Object obj) {
            return obj instanceof ColorPoint;
        }
    }
    \end{minted}
    
    \textbf{Пояснение:}
    \begin{enumerate}
        \item Метод \texttt{canEqual} используется для проверки, является ли объект экземпляром текущего класса.
        \item Это помогает сохранить симметричность и транзитивность при наследовании.
    \end{enumerate}
    
\end{enumerate}