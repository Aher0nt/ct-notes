Синтаксис\texttt{Java} — это набор правил, которые определяют, как должны быть структурированы программы на \texttt{Java}. Синтаксис включает в себя различные элементы, такие как типы данных, операции, операторы, структуру исходного кода и многое другое.

\subsection{Типы данных}
Типы данных в \texttt{Java} можно разделить на две основные категории: примитивные типы и ссылочные (объектные) типы.

\subsubsection{Примитивные типы данных}
Примитивные типы данных являются базовыми элементами, используемыми для хранения простых значений, таких как числа и символы. В \texttt{Java} существует 8 примитивных типов:

\begin{itemize}
    \item \textbf{Целочисленные типы}:
    \begin{itemize}
        \item \texttt{byte} (8 бит): хранит значения от $-128$ до $127$.
        \item \texttt{short} (16 бит): хранит значения от $-32,768$ до $32,767$.
        \item \texttt{int} (32 бита): хранит значения от $-2^{31}$ до $2^{31-1}$.
        \item \texttt{long} (64 бита): хранит значения от $-2^{63}$ до $2^{63-1}$.
    \end{itemize}

    \item \textbf{Числа с плавающей точкой}:
    \begin{itemize}
        \item \texttt{float} (32 бита): используется для хранения чисел с плавающей точкой одинарной точности.
        \item \texttt{double} (64 бита): используется для хранения чисел с плавающей точкой двойной точности.
    \end{itemize}

    \item \textbf{Символьный тип}:
    \begin{itemize}
        \item \texttt{char} (16 бит): хранит один символ в кодировке \texttt{UTF-16}.
    \end{itemize}

    \item \textbf{Логический тип}:
    \begin{itemize}
        \item \texttt{boolean}: может принимать значения \texttt{true} или \texttt{false}.
    \end{itemize}

    \item \textbf{Обёртки примитивных типов}: \\
    \texttt{Java} предоставляет классы-обёртки для каждого примитивного типа, которые позволяют работать с примитивами как с объектами. Например, для типа \texttt{int} существует класс \texttt{Integer}, для типа \texttt{double} — класс \texttt{Double} и т. д. Это полезно, когда нужно использовать примитивные типы в коллекциях, так как коллекции работают только с объектами.
\end{itemize}

\subsubsection{Массивы}
Массивы представляют собой структуру данных, которая хранит элементы одного типа в фиксированном размере.

\begin{itemize}
    \item \textbf{Массивы как объекты} \\
    Массивы в Java являются объектами, и для них можно вызвать методы класса \texttt{Object}. Например, можно использовать \texttt{array.length} для получения длины массива.
    
    \item \textbf{Ковариантность} \\
    Массивы в \texttt{Java} ковариантны, то есть массив типа \texttt{SuperType[]} может ссылаться на массив типа \texttt{SubType[]}, если \texttt{SubType} является подклассом \texttt{SuperType}.
    
    \item \textbf{Reification (Восстановление типов)} \\
    В \texttt{Java} массивы сохраняют информацию о типе своих элементов во время выполнения (\texttt{reification}). Это означает, что \texttt{Java} может проверять тип элементов массива в рантайме.
\end{itemize}


\subsection{Операции}
Операции в \texttt{Java} — это действия, которые могут выполняться над операндами. Операторы, используемые в этих операциях, имеют определённые приоритеты и ассоциативность.

\subsubsection{Приоритеты}
Приоритет операторов определяет порядок их выполнения в выражении. Операторы с более высоким приоритетом выполняются раньше операторов с более низким приоритетом.

\subsubsection{Ассоциативность}
Ассоциативность определяет порядок выполнения операторов с одинаковым приоритетом: слева направо или справа налево.


\subsection{Структура исходного кода}
Программа на Java состоит из классов и методов, которые могут содержать поля, конструкторы и блоки инициализации.

\subsubsection{Заголовок}
Каждый файл начинается с заголовка, включающего объявление пакета и импорт необходимых классов.
\begin{minted}{java}
package com.example;

import java.util.List;
\end{minted}

\subsubsection{Классы}
Классы — это основные строительные блоки в \texttt{Java}. Они могут содержать поля, методы, конструкторы и внутренние классы.
\begin{minted}{java}
public class MyClass {
    // Поля, методы и конструкторы
}
\end{minted}

\subsubsection{Интерфейсы}
Интерфейсы определяют контракты для реализации другими классами.
\begin{minted}{java}
public interface MyInterface {
    void doSomething();
}
\end{minted}

\subsubsection{Поля}
Поля (переменные класса) используются для хранения состояния объекта.
\begin{minted}{java}
public class MyClass {
    private int value;
}
\end{minted}

\subsubsection{Конструкторы}
Конструкторы — это методы, которые вызываются при создании объекта.
\begin{minted}{java}
public MyClass(int value) {
    this.value = value;
}
\end{minted}

\subsubsection{Методы}
Методы определяют поведение объекта.
\begin{minted}{java}
public void display() {
    System.out.println("Value: " + value);
}
\end{minted}
