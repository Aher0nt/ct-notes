\subsection{Класс и основная функция}
Код на Java начинается с создания класса.\newline
Название класса файла должно совпадать с названием файла. \par
Используем \texttt{public class FileName}, чтобы создать класс.
\begin{minted}[linenos, frame=leftline, tabsize=4, fontsize=\ttfamily\small, framesep=4mm, numbersep=4pt]{java}
// В файле "HelloWorld.java" создадим класс "HelloWorld"
public class HelloWorld {}
\end{minted}
\vspace{0.3cm}

Чтобы можно было запускать программу напрямую (как основной файл), в классе расположим функцию \texttt{main()}. Она задается как \texttt{public static void}.
\begin{minted}[linenos, frame=leftline, tabsize=4, fontsize=\ttfamily\small, framesep=4mm, numbersep=4pt]{java}
// В классе "HellowWorld" расположим функцию "main"
public class HelloWorld {
    public static void main() {}
}
\end{minted}
\vspace{0.3cm}

Принятно передавать в \texttt{main()} аргументы, введенные при запуске. Это делается через массив строк, названный \texttt{args}.
\begin{minted}[linenos, frame=leftline, tabsize=4, fontsize=\ttfamily\small, framesep=4mm, numbersep=4pt]{java}
public class HelloWorld {
    public static void main(String[] args) {}
}
\end{minted}
\vspace{0.3cm}

\subsection{Переменные}
\begin{itemize}
    \item \textbf{Основные типы данных в Java}
    \begin{itemize}
        \item \texttt{boolean}: хранит значение true или false.
        \item \texttt{byte}: целое число от $-2^{7}$ до $2^{7} - 1$ (1 байт).
        \item \texttt{short}: целое число от $-2^{15}$ до $2^{15} - 1$ (2 байт).
        \item \texttt{int}: целое число от $-2^{31}$ до $2^{31} - 1$ (4 байт).
        \item \texttt{long}: целое число от $-2^{63}$ до $2^{63} - 1$ (8 байт).
        \item \texttt{float}: число с плавающей точкой (4 байта).
        \item \texttt{double}: число с плавающей точкой двойной точности (8 байт).
        \item \texttt{char}: символ Unicode (2 байта).
    \end{itemize}
    \item \textbf{Ссылочные типы данных}
    \begin{itemize}
        \item \texttt{variable[]}: массив, где \texttt{variable} — тип данных
        \item \texttt{String}: строка
    \end{itemize}
\end{itemize}


Переменные обявляются в помощью конструкции \texttt{variable: variableName;}.\\
Переменную можно сразу инициализировать, придав ей значение.
\begin{minted}[linenos, frame=leftline, tabsize=4, fontsize=\ttfamily\small, framesep=4mm, numbersep=4pt]{java}
public class HelloWorld {
    public static void main(String[] args) {
        boolean firstBoolVariable;
        boolean secondBoolVariable = true;
        int intVariable = 123;
        String stringVariable = "HelloWorld";
    }
}
\end{minted}

\subsection{Стандартный поток ввода и вывода}
Стандартный поток осуществляется через класс \texttt{System}.
\begin{itemize}
    \item Через статические поле \texttt{out} производится поток вывода.
    \item Через статические поле \texttt{err} производится поток ввода ошибок.
    \begin{minted}[linenos, frame=leftline, tabsize=4, fontsize=\ttfamily\small, framesep=4mm, numbersep=4pt]{java}
public class HelloWorld {
    public static void main(String[] args) {
        System.out.println("Это стандартное сообщение.");
        System.err.println("Это сообщение об ошибке.");
    }
}
    \end{minted}
    \vspace{0.4cm}

    \item Через статические поле \texttt{in} производится поток ввода.

    Однако чтобы им воспользоваться, необходимо использовать специальный класс \texttt{Scanner}.
    \begin{minted}[linenos, frame=leftline, tabsize=4, fontsize=\ttfamily\small, framesep=4mm, numbersep=4pt]{java}
import java.util.Scanner;
public class HelloWorld {
    public static void main(String[] args) {
        Scanner scanner = new Scanner(System.in);
        System.out.println("Введите имя: ");
        String name = scanner.nextLine();
        System.out.println("Здравствуйте, " + name);
    }
}
    \end{minted}
    \vspace{0.4cm}
\end{itemize}



\subsection{Условные конструкции}
В Java условные конструкции работают вполне стандартно.\par
Условная конструкция начинается с \texttt{if}, потом круглых скобочках передается условие — некоторое выражение, возвращающее булевое значение (\texttt{true} или \texttt{false}).\newline
В фигурных скобочках же указывается тело условной конструкции — то, что будет выполнено при значении true.\par
С помощью конструкции \texttt{else if} можно прописать дополнительное условие, если в прошлом булевое выражение вернуло \texttt{false}.\newline
Так же есть возможность прописать отдельное тело, если ни одно условие не вернуло true, это делается с помощью \texttt{else}.

\begin{minted}[linenos, frame=leftline, tabsize=4, fontsize=\ttfamily\small, framesep=4mm, numbersep=4pt]{java}
// Получим количество введенных при запуске агрументов с помощью метода .length

public class HelloWorld {
    public static void main(String[] args) {
        int numOfArgs = args.length 
        
        if (args.length == 0) {
            System.out.println("Hello, World!");
        } else if (args.length == 1) {
            System.out.println("Hello, " + args[0]);
        } else {
            System.err.println("Expected zero or one argument, got: " + args.length);
        }
    }
}
\end{minted}

\vspace{0.4cm}

\subsection{Циклы}
В Java существует несколько видов циклов. Они вполне стандартны и взаимозаменяемы, обычно есть в любом языке.
\begin{itemize}
    \item Цикл \texttt{for}: Этот цикл используется, когда известно количество итераций заранее.
    
    В круглых скобочках настроивается работа цикла.
    \begin{itemize}
        \item Сначала инициализируется переменная, которая будет выступать в качестве счетчика.
        \item Далее указывается условие, при котором цикл продолжает запускать итерации.
        \item И в конце указывается то, как счетчик будет меняться с каждой итерацией.
    \end{itemize}

    В фигурных скобочках размещается тело цикла. Код внутри будет запускаться каждую итерацию.

    \begin{minted}[linenos, frame=leftline, tabsize=4, fontsize=\ttfamily\small, framesep=4mm, numbersep=4pt]{java}
// Данный цикл будет работать 10 раз, выводя номер итерации.

for (int i = 0; i < 10; i++) {
            System.out.println(i);
}

// Пример бесконечного цикла if

if (;;) {
    System.out.println("Беспокнечный цикл");
}
    \end{minted}
    \vspace{0.4cm}

    \item Цикл \texttt{while}: Этот цикл выполняется до тех пор, пока условие, указанное в круглых скобочках, истинно.\\
    Он проверяет условие перед каждой итерацией.\par
    Как и у цикла \texttt{if}, в фигурных скобочках размещается тело цикла. Код внутри будет запускаться каждую итерацию.

    \begin{minted}[linenos, frame=leftline, tabsize=4, fontsize=\ttfamily\small, framesep=4mm, numbersep=4pt]{java}
// Данный цикл будет работать пока i меньше 64, выводя степени двойки до 64.

int i = 1;
while (i < 64) {
    System.out.println(i);
    i = i * 2;
}

// Пример бесконечного цикла while

while (true) {
    System.out.println("Беспокнечный цикл");
}
    \end{minted}
    \vspace{0.4cm}


    \item Цикл \texttt{do...while}: Этот цикл похож на \texttt{while}, но условие проверяется после выполнения тела цикла, что гарантирует хотя бы одно выполнение действий.

    \begin{minted}[linenos, frame=leftline, tabsize=4, fontsize=\ttfamily\small, framesep=4mm, numbersep=4pt]{java}
// Данный цикл продолжает итерации пока i меньше 2
// Несмотря на то, что i изначально задана как 2, цикл произведет первую итерацию.

int i = 2
do {
    System.out.println(i);
    i = i * 2;
} while (i < 2);
    \end{minted}
    \vspace{0.4cm}

    
    \item Цикл \texttt{for-each}: Этот цикл используется для перебора элементов коллекции или массива.\par
    В круглых скобочках указыватся тип элементов и сам объект, который нужно перебрать.

    \begin{minted}[linenos, frame=leftline, tabsize=4, fontsize=\ttfamily\small, framesep=4mm, numbersep=4pt]{java}
// Этот цикл переберет все числа в массиве numbers.

int[] numbers = {1, 2, 3, 4, 5};
for (int number : numbers) {
    System.out.println(number);
}
    \end{minted}
\end{itemize}


\subsection{Функции}
В Java функции называются методами. Метод — это блок кода, который выполняет определенную задачу и может быть вызван по имени. Методы позволяют структурировать и переиспользовать код.

\begin{enumerate}
    \item Объявление метода: Методы объявляются внутри классов.
    Если метод ничего не возвращает, используется ключевое слово \texttt{void}.

    \begin{minted}[linenos, frame=leftline, tabsize=4, fontsize=\ttfamily\small, framesep=4mm, numbersep=4pt]{java}
public static void printMessage() {
    System.out.println("Hello, World!");
}
    \end{minted}
    \vspace{0.4cm}


    \item Параметры метода: Методы могут принимать параметры, которые передаются при вызове метода.
    
    \begin{minted}[linenos, frame=leftline, tabsize=4, fontsize=\ttfamily\small, framesep=4mm, numbersep=4pt]{java}
public void greet(String name) {
    System.out.println("Hello, " + name);
}
    \end{minted}
    \vspace{0.4cm}


    \item Возвращаемый тип: Если метод возвращает значение, необходимо обозначить тип возвращаемых данных во время объявления метода. 

    \begin{minted}[linenos, frame=leftline, tabsize=4, fontsize=\ttfamily\small, framesep=4mm, numbersep=4pt]{java}
public int sum(int a, int b) {
    return a + b;
}
    \end{minted}

    Метод возвращает \texttt{int}, поэтому и объявлен как \texttt{public int sum()}.    
\end{enumerate}