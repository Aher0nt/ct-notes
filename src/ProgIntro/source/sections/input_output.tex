\subsection{Исключения}
\begin{itemize}
    \item \textbf{try-catch} \par
    Исключение — это событие, которое происходит во время выполнения программы и нарушает нормальный поток её инструкций. \newline
    Когда в методе возникает ошибка, метод создает объект исключения и передает его системе выполнения. \par
    Конструкция \texttt{try-catch} используется для обработки исключений в Java.\par
    Код, который может вызвать исключение, помещается в блок \texttt{try}. Если возникает исключение, оно перехватывается блоком \texttt{catch}, где можно обработать ошибку.
    \begin{minted}[linenos, frame=leftline, tabsize=4, fontsize=\ttfamily\small, framesep=4mm, numbersep=4pt]{java}
try {
    // Код, который может вызвать исключение
    int result = 10 / 0;
} catch (ArithmeticException e) {
    // Обработка исключения
    System.out.println("Деление на ноль запрещено!");
}
    \end{minted}
    Важно правильно определять тип исключения в блоке catch.\newline
    Если исключение не будет поймано, программа завершится с ошибкой.

    \item \textbf{Проверяемые исключения, throws} \par
    Проверяемые исключения \texttt{(checked exceptions)} должны быть либо обработаны в блоке \texttt{try-catch}, либо объявлены в сигнатуре метода с помощью ключевого слова \texttt{throws}. \par
    Ключевое слово \texttt{throws} в Java используется для указания того, что метод может выбросить одно или несколько исключений. Это позволяет информировать вызывающий метод о возможных исключениях, чтобы они могли быть обработаны соответствующим образом. \par
    \begin{minted}[linenos, frame=leftline, tabsize=4, fontsize=\ttfamily\small, framesep=4mm, numbersep=4pt]{java}
public void readFile(String fileName) throws IOException {
    FileReader file = new FileReader(fileName);
    BufferedReader fileInput = new BufferedReader(file);
    fileInput.close();
}
    \end{minted}
    В данном случае метод readFile может выбросить IOException, это ошибку нужно обработать при вызове readFile. \par
    Все исключения, кроме наследников \texttt{RuntimeException}, являются проверяемыми.

    \item \textbf{Обработка исключений} \par
    Обработка исключений позволяет программе продолжать выполнение после возникновения ошибки. Это достигается путем перехвата и обработки исключений, что предотвращает завершение программы.
    
    \begin{itemize}
        \item \textbf{Несколько catch-блоков} \par
        Можно использовать несколько блоков \texttt{catch} для обработки различных типов исключений.
        \begin{minted}[linenos, frame=leftline, tabsize=4, fontsize=\ttfamily\small, framesep=4mm, numbersep=4pt]{java}
try {
    int[] numbers = {1, 2, 3};
    System.out.println(numbers[10]);
} catch (ArrayIndexOutOfBoundsException e) {
    System.out.println("Ошибка: индекс вне границ массива");
} catch (Exception e) {
    System.out.println("Общая ошибка");
}
        \end{minted}
        Блоки \texttt{catch} проверяются сверху вниз, поэтому более специфичные исключения должны быть выше.

        \item \textbf{Сообщения об ошибках} \par
        Сообщения об ошибках помогают понять причину исключения.
        \begin{minted}[linenos, frame=leftline, tabsize=4, fontsize=\ttfamily\small, framesep=4mm, numbersep=4pt]{java}
try {
    int result = 10 / 0;
} catch (ArithmeticException e) {
    System.err.println("Ошибка: " + e.getMessage());
}
        \end{minted}
        Можно вручную писать текст ошибки, и выводить через стандартный поток вывода ошибок или ипользовать \texttt{e.getMessage()} для получения подробного сообщения об ошибке.

        \item \textbf{Стек исполнения} \par
        Стек исполнения (stack trace) показывает последовательность вызовов методов, которые привели к исключению. Используйте метод \texttt{e.printStackTrace();}.
        \begin{minted}[linenos, frame=leftline, tabsize=4, fontsize=\ttfamily\small, framesep=4mm, numbersep=4pt]{java}
try {
    int result = 10 / 0;
} catch (ArithmeticException e) {
    e.printStackTrace();
}
        \end{minted}
        Полезно для отладки и понимания, где произошло исключение.
    \end{itemize}
\end{itemize}

\subsection{Ресурсы}
Ресурсы в Java — это объекты, которые должны быть закрыты после использования. Примеры включают файлы, сетевые соединения и базы данных. Закрытие ресурсов важно для предотвращения утечек ресурсов.

\begin{itemize}
    \item \textbf{Закрытие и утечка ресурсов} \par
    Если ресурсы не закрываются, это может привести к утечке ресурсов. \par Утечка ресурсов происходит, когда программа не освобождает ресурсы после их использования, что может привести к исчерпанию доступных ресурсов, снижению производительности и ошибкам.

    \item \textbf{try-catch-finally} \par
    Конструкция \texttt{try-catch-finally} используется для обработки исключений и гарантирует выполнение определенного кода независимо от того, произошло исключение или нет.

    \begin{itemize}
        \item \texttt{try}: Блок, в котором выполняется код, который может вызвать исключение.
        \item \texttt{catch}: Блок, который обрабатывает исключения, возникшие в блоке \texttt{try}.
        \item \texttt{finally}: Блок, который выполняется в любом случае, независимо от того, было ли исключение или нет. 
    \end{itemize}
    \begin{minted}[linenos, frame=leftline, tabsize=4, fontsize=\ttfamily\small, framesep=4mm, numbersep=4pt]{java}
try {
    // Код, который может вызвать исключение
} catch (ExceptionType e) {
    // Обработка исключения
} finally {
    // Код, который выполняется в любом случае (например закрытие ресурсов)
}
    \end{minted}
    Обычно используется для освобождения ресурсов.

    \item \textbf{Блок использования ресурса} \par
    С Java 7 был введен \texttt{try-with-resources} — это специальная конструкция \texttt{try}, которая автоматически закрывает ресурсы по завершении блока.
    \begin{minted}[linenos, frame=leftline, tabsize=4, fontsize=\ttfamily\small, framesep=4mm, numbersep=4pt]{java}
try (BufferedReader br = new BufferedReader(new FileReader("file.txt"))) {
    // Использование ресурса
} catch (IOException e) {
    // Обработка исключения
}
    \end{minted}
    Ресурс должен реализовывать интерфейс \texttt{AutoCloseable}.

    \item \textbf{Использование нескольких ресурсов одновременно} \par
    Можно использовать несколько ресурсов в одном блоке try-with-resources.
    \begin{minted}[linenos, frame=leftline, tabsize=4, fontsize=\ttfamily\small, framesep=4mm, numbersep=4pt]{java}
try (FileReader file1 = new FileReader("file1.txt");
     FileReader file2 = new FileReader("file2.txt")) {
    // работа с файлами
} catch (IOException e) {
    e.printStackTrace();
}
    \end{minted}
    Все ресурсы будут закрыты в обратном порядке их объявления.
\end{itemize}

\subsection{Кодировки}
Кодировки играют важную роль при работе с текстовыми данными в Java, особенно при чтении и записи файлов. Давайте рассмотрим основные аспекты работы с кодировками. 

\begin{itemize}
    \item \textbf{Кодировка по-умолчанию} \par
    При работе с файлами в Java, кодировка по умолчанию зависит от операционной системы. Например, на Windows это может быть \texttt{Cp1251} для русского языка, а на Unix-подобных системах — \texttt{UTF-8}. Однако, начиная с Java 18, кодировка по умолчанию для всех платформ была изменена на \texttt{UTF-8}.
    
    \item \textbf{Явное указание кодировки}
    Чтобы явно указать кодировку при работе с файлами, можно использовать соответствующие конструкторы и методы классов ввода-вывода. \par
    Указание кодировки помогает избежать проблем с несовместимостью. \par
\end{itemize}

\subsection{Readers}
\begin{itemize}
    \item \textbf{Reader} \par 
    \texttt{Reader} — это абстрактный класс, который используется для чтения потоков символов. \par
    Он является родительским классом для многих других классов, таких как \texttt{FileReader}, \texttt{BufferedReader} и \texttt{InputStreamReader}. \par 
    Основные методы включают \texttt{read()}, \texttt{close()}, \texttt{mark()}, \texttt{reset()} и \texttt{skip()}.
    \begin{minted}[linenos, frame=leftline, tabsize=4, fontsize=\ttfamily\small, framesep=4mm, numbersep=4pt]{java}
import java.io.Reader;
import java.io.StringReader;

public class ReaderExample {
    public static void main(String[] args) {
        String data = "Hello, World!";
        Reader reader = new StringReader(data);
        try {
            int character;
            while ((character = reader.read()) != -1) {
                System.out.print((char) character);
            }
        } catch (IOException e) {
            e.printStackTrace();
        } finally {
            try {
                reader.close();
            } catch (IOException e) {
                e.printStackTrace();
            }
        }
    }
}
    \end{minted}
    \texttt{Reader} сам по себе не может быть использован напрямую, так как он абстрактный. Необходимо использовать его подклассы. \par
    Важно закрывать Reader после использования, чтобы освободить ресурсы.

    \item \textbf{FileReader} \par
    \texttt{FileReader} используется для чтения текстовых файлов. Он является подклассом \texttt{InputStreamReader} и наследует все его методы. Предназначен для упрощения чтения файлов.
    \begin{minted}[linenos, frame=leftline, tabsize=4, fontsize=\ttfamily\small, framesep=4mm, numbersep=4pt]{java}
import java.io.FileReader;
import java.io.IOException;

public class FileReaderExample {
    public static void main(String[] args) {
        try (FileReader reader = new FileReader("example.txt")) {
            int character;
            while ((character = reader.read()) != -1) {
                System.out.print((char) character);
            }
        } catch (IOException e) {
            e.printStackTrace();
        }
    }
}
    \end{minted}
    Использует кодировку по умолчанию для вашей системы.\newline
    Если вам нужно использовать другую кодировку, \\ лучше использовать \texttt{InputStreamReader} с указанием кодировки. \par
    \texttt{FileReader} бросает \texttt{FileNotFoundException}, если файл не существует,\\ и \texttt{IOException} для других ошибок ввода-вывода.

    \item \textbf{BufferedReader} \par
    Используется для более эффективного чтения текста из входного потока символов, буферизуя символы, чтобы обеспечить эффективное считывание строк и массивов символов. \par
    \begin{minted}[linenos, frame=leftline, tabsize=4, fontsize=\ttfamily\small, framesep=4mm, numbersep=4pt]{java}
import java.io.BufferedReader;
import java.io.FileReader;
import java.io.IOException;

public class BufferedReaderExample {
    public static void main(String[] args) {
        try (BufferedReader reader = new BufferedReader(
            new FileReader("example.txt")
        )) {
            String line;
            while ((line = reader.readLine()) != null) {
                System.out.println(line);
            }
        } catch (IOException e) {
            e.printStackTrace();
        }
    }
}
    \end{minted}
    Значительно улучшает производительность при чтении больших файлов, так как уменьшает количество обращений к диску. \par
    По умолчанию размер буфера составляет 8192 символа, но его можно изменить, указав второй параметр в конструкторе.
    
    \item \textbf{InputStreamReader} \par
    \texttt{InputStreamReader} преобразует байтовый поток в поток символов, что позволяет читать текстовые данные из байтовых потоков. \par
    Он является мостом между байтовыми и символьными потоками.
    \begin{minted}[linenos, frame=leftline, tabsize=4, fontsize=\ttfamily\small, framesep=4mm, numbersep=4pt]{java}
import java.io.InputStreamReader;
import java.io.FileInputStream;
import java.io.IOException;

public class InputStreamReaderExample {
    public static void main(String[] args) {
        try (InputStreamReader reader = new InputStreamReader(
            new FileInputStream("example.txt"), "UTF-8"
        )) {
            int character;
            while ((character = reader.read()) != -1) {
                System.out.print((char) character);
            }
        } catch (IOException e) {
            e.printStackTrace();
        }
    }
}
    \end{minted}
    \texttt{InputStreamReader} позволяет указать кодировку, что делает его более гибким по сравнению с \texttt{FileReader}.\par
    Важно правильно указывать кодировку
    
    \item \textbf{InputStream} \par
    \texttt{InputStream} — это абстрактный класс для чтения байтов из потока. Он является базовым классом для всех классов, которые читают байты из потока. \par
    \texttt{InputStream} сам по себе не может быть использован напрямую, так как он абстрактный. Необходимо использовать его подклассы.
\end{itemize}

\subsection{Writers}
\begin{itemize}
    \item \textbf{Writer} \par
    \texttt{Writer} — это абстрактный класс в пакете \texttt{java.io}, который представляет поток символов. Он является родительским классом для всех классов, которые пишут символы в выходной поток. \par
    Основные методы включают:
    \begin{itemize}
        
    \item \texttt{write(int c)}: записывает один символ.
    \item \texttt{write(char[] cbuf, int off, int len)}: записывает часть массива символов.
    \item \texttt{write(String str, int off, int len)}: записывает часть строки.
    \item \texttt{flush()}: очищает поток.
    \item \texttt{close()}: закрывает поток.
    \end{itemize}


    \item \textbf{FileWriter} \par
    \texttt{FileWriter} — это подкласс \texttt{Writer}, который используется для записи текстовых данных в файл. Он использует кодировку символов по умолчанию для платформы, если не указано иное. \par
    Основные методы:
    \begin{enumerate}
        \item \texttt{FileWriter(String fileName): создает объект FileWriter}, связанный с файлом.
        \item \texttt{FileWriter(String fileName, boolean append)}: создает объект FileWriter, связанный с файлом, с возможностью добавления данных в конец файла.
        \item Методы записи аналогичны методам \texttt{Writer}.
    \end{enumerate}
    \begin{minted}[linenos, frame=leftline, tabsize=4, fontsize=\ttfamily\small, framesep=4mm, numbersep=4pt]{java}
FileWriter writer = new FileWriter("output.txt");
writer.write("Hello, World!");
writer.close();
    \end{minted}

    \item \textbf{BufferedWriter} \par
    \texttt{BufferedWriter} — это подкласс \texttt{Writer}, который буферизует символы для повышения эффективности записи. Он оборачивает другой объект \texttt{Writer} и добавляет буферизацию. \par
    Основные методы.
    \begin{enumerate}
        \item \texttt{BufferedWriter(Writer out)}: создает буферизированный поток.
        \item \texttt{BufferedWriter(Writer out, int sz)}: создает буферизированный поток с указанным размером буфера.
        \item Методы записи аналогичны методам \texttt{Writer}.
    \end{enumerate}
    \begin{minted}[linenos, frame=leftline, tabsize=4, fontsize=\ttfamily\small, framesep=4mm, numbersep=4pt]{java}
BufferedWriter bufferedWriter = new BufferedWriter(new FileWriter("output.txt"));
bufferedWriter.write("Hello, Buffered World!");
bufferedWriter.close();
    \end{minted}

    \item \textbf{OutputStreamWriter} \par
    \texttt{OutputStreamWriter} — это мост между потоками символов и байтов. Он преобразует символы в байты с использованием указанной кодировки.\par
    Основные методы:
    \begin{enumerate}
        \item \texttt{OutputStreamWriter(OutputStream out)}: создает объект \texttt{OutputStreamWriter}, связанный с выходным потоком байтов.
        \item \texttt{OutputStreamWriter(OutputStream out, String charsetName)}: создает объект \texttt{OutputStreamWriter} с указанной кодировкой.
        \item Методы записи аналогичны методам \texttt{Writer}.
    \end{enumerate}
    \begin{minted}[linenos, frame=leftline, tabsize=4, fontsize=\ttfamily\small, framesep=4mm, numbersep=4pt]{java}
OutputStreamWriter outputStreamWriter = new OutputStreamWriter(
    new FileOutputStream("output.txt"), "UTF-8"
);
outputStreamWriter.write("Hello, UTF-8 World!");
outputStreamWriter.close();
    \end{minted}
    
    \item \textbf{OutputStream} \par
    \texttt{OutputStream} — это абстрактный класс, представляющий поток байтов. Он является суперклассом для всех классов, которые пишут байты в выходной поток.

    \item \textbf{PrintWriter} \par
    PrintWriter — это подкласс Writer, который предоставляет удобные методы для форматированной записи данных в текстовый поток. Он поддерживает автоматическую очистку буфера и может быть обернут вокруг других объектов Writer.\par
    Основные методы:
    \begin{enumerate}
        \item PrintWriter(Writer out): создает объект PrintWriter, связанный с другим объектом Writer.
        \item \texttt{PrintWriter(OutputStream out)}: создает объект \texttt{PrintWriter}, связанный с выходным потоком байтов.
        \item \texttt{print(...)}: методы для записи различных типов данных.
        \item \texttt{printf(...)}: методы для форматированной записи данных.
        \item \texttt{println(...)}: методы для записи данных с новой строки.
    \end{enumerate}
    \begin{minted}[linenos, frame=leftline, tabsize=4, fontsize=\ttfamily\small, framesep=4mm, numbersep=4pt]{java}
PrintWriter printWriter = new PrintWriter(new FileWriter("output.txt"));
printWriter.println("Hello, PrintWriter!");
printWriter.close();
    \end{minted}
\end{itemize}