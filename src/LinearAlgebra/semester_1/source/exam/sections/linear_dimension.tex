\subsection{Основные определения.}
В этом разделе мы будем рассматривать линейные пространства над $\mathbb C$ и иногда $\mathbb R$.  Обозначать над чем мы будем $K$.
\subsubsection{Линейная оболочка, линейная независимость векторов.}
  Говорят, что вектор $u$ является \deff{линейной комбинацией} векторов $(v_1;v_2;\ldots;v_n)$, если  $\exists\alpha_1;\alpha_2;\ldots;\alpha_n\in K u=\sum\limits_{i=1}^n\alpha_i\cdot v_i$.

  Если все \(\lambda_k = 0\), то линейная комбинация называется \deff{тривиальной}

Система векторов \(v_1, \ldots, v_m \in V\) называется \deff{линейной независимой}, если любая нулевая линейная комбинация тривиальна \(\defLeftrightarrow \sum\limits_{k = 1}^{m}\lambda_k v_k = 0 \Leftrightarrow \forall k \in \{1, \ldots, m\}: \lambda_k = 0\)

В противном случае, система векторов называется \deff{линейно зависимой}, т.е. \(\exists\) набор \(\lambda_1, \ldots, \lambda_m\) не все нули таких, что \(\sum\limits_{k = 1}^{m} \lambda_k v_k = 0\). 

\subsubsection{Теорема о линейно независимых системах векторов}
\thmm{Теорема}

\begin{enumerate}
    \item \(v_1, \ldots, v_m\) -линейно зависима \(\Leftrightarrow\) по крайней мере один из векторов --- это линейная комбинация остальных

    \item Если некоторая подсистема системы векторов \(v_1, \ldots, v_m\) - линейно зависима, то система векторов \(v_1, \ldots, v_m\) --- линейно зависима
    \item
          \(\begin{rcases*}
              v_1, \ldots, v_m - \text{линейно независима} \\
              v_1, \ldots, v_{m + 1} - \text{линейно зависима}
          \end{rcases*} \Rightarrow\) \(v_{m + 1}\) --- линейная комбинация \(v_1, \ldots, v_m\)
\end{enumerate}

Доказательство очевидно

\textbf{Следствия:}

\begin{enumerate}
    \item Если система линейно независима, то любая подсистема линейно независима

    \item Если система содержит \(0\) вектор, либо пару пропорциональных векторов, то система линейно зависима
\end{enumerate}



\subsubsection{Теорема о прополке.}

Любую систему векторов \(v_1, \ldots, v_m\), в которой хотя бы один из векторов ненулевой, можно заменить на линейно независимую систему векторов \(v_{j_1}, \ldots, v_{j_k}\) с сохранением линейной оболочки. \(\spann (v_1, \ldots, v_m) = \spann (v_{j_1}, \ldots, v_{j_k})\)

\begin{enumerate}
    \item[] \prooff{}
    Пусть \(s_0 = 0, s_1 = \spann (v_1), \ldots, s_m = \spann (v_1, \ldots, v_m)\)

Тогда \(s_0 \subset s_1 \subset \ldots \subset s_m \subset V\)

Идём от \(j = m\) до \(j = 2\)

Если \(s_{j - 1} = s_j\), то \(v_j\) удаляем. При этом \(\spann (v_1, \ldots, v_j) = \spann (v_1, \ldots, v_{j - 1})\) сохраняется

Если \(s_{j - 1} \subset s_j\), то \(v_j \notin s_{j - 1}\), т.е. \(v_j\) --- не является линейной комбинацией \(v_1, \ldots, v_{j - 1}\)

Продолжая так делать, получим, что никакой вектор из полученных не является линейной комбинацией других, то есть итоговое подмножество линейно независимо

В результате получается цепочка строго вложенных подмножеств \(s_0 \subset s_{j_1} \subset \ldots \subset s_{j_k} \subset s_m \subset V \)

\(\Rightarrow s_m = \spann (v_{j_1}, \ldots, v_{j_k})\) \hfill Q.E.D.
\end{enumerate}

\subsection{Порождающие системы, базис, размерность и т.п.}

\subsubsection{Порождающая система векторов, конечномерные пространства.}

Система векторов \(v_1, \ldots, v_m \in V\) называется порождающей (полной), если любой вектор линейного пространства \(V\) раскладывается по этим векторам, т.е. является линейной комбинацией \(v_1, \ldots, v_m\). \(V = \spann (v_1, \ldots, v_m)\)

Если число \(v_1, \ldots, v_m\) конечно, то линейное пространство называется конечномерным.

\subsubsection{Базис. Теорема об эквивалентных условиях для базиса.}

\thmm{Теорема}

Следующие утверждения равносильны:

\begin{enumerate}
    \item \(v_1, \ldots, v_n \in V\) --- линейно независимая и порождающаяся система

    \item \(v_1, \ldots, v_n \in V\) --- линейно независимая система и максимальная по числу элементов

    \item \(v_1, \ldots, v_n \in V\) --- порождающая система и минимальная по числу элементов
\end{enumerate}
 Доказательство очевидно.
 
 Если система \(v_1, \ldots, v_n \in V\) удовлетворяет условиям теоремы, то она называется \deff{базисом} пространства \(V\).

\subsubsection{Размерность пространства.}
 Количество векторов \(n = \dim V = \) \deff{размерность линейного пространства} \(= \max\) возможное число линейно независимых векторов \(= \min\) число в порождающей системе векторов

 \subsubsection{Теорема о дополнении любой независимой системы до базиса и о порождающей системе векторов.}
 

\begin{enumerate}
    \item \(\forall\) линейно независимую систему векторов в \(V\) можно дополнить до базиса пространства \(V\)

    \item из любой порождающей системы пространства \(V\) можно выделить базис пространства \(V\)
\end{enumerate}

Доказательство очевидно, спасибо прополке.

\subsection{Координаты, изоморфзим и все об этом.}
\subsubsection{Координаты вектора и их единственность.}

\(\forall x \in V: x = \sum\limits_{i = 1}^{n} x_i l_i\), где \(l = (l_1, \ldots, l_n)\) --- базис в \(V\) (порождающей системы)

\(x_i \in K\) --- координаты вектора \(x\) относительно базиса \(l\)

\(x \in V \longrightarrow x =
\begin{pmatrix}
    x_{1}  \\
    \vdots \\
    x_{n}
\end{pmatrix} \in K^n\), где \(
\begin{pmatrix}
    x_{1}  \\
    \vdots \\
    x_{n}
\end{pmatrix}
\) --- координатный столбец.

\textbf{Утверждение}

\(\forall x \in V\) координаты относительно базиса \(e\) определяется единственным образом

Доказательство очевидно от противного.

Несложно заметить, что отображение между $V$ и $K^n$, которое по вектору выдаёт его координаты --- биекция.
\subsubsection{Изоморфизм линейных пространств и его свойства.}

\(V_1, V_2\) --- линейные пространства называются изоморфными (\(V_1 \cong V_2\)), если между \(V_1\) и \(V_2\) существует биекция и сохраняется линейность, т.е.
\begin{gather*}
    x \in V_1 \longleftrightarrow x' \in V_2 \\
    y \in V_1 \longleftrightarrow y' \in V_2 \\
    \forall \lambda \in K: x + \lambda y \in V_1 \longleftrightarrow x' + \lambda y' \in V_2
\end{gather*}

\textbf{Свойства изоморфизма}

\begin{enumerate}
    \item \(0 \in V \longrightarrow 0' \in V'\)

          Доказательство

          \(\forall \lambda \in K: \lambda x \longleftrightarrow \lambda x'\)

          Пусть \(\lambda = 0\), тогда \(0 = 0 \cdot x \longleftrightarrow 0 \cdot x' = 0'\)
          \hfill Q.E.D.

    \item \(\forall x \in V \longleftrightarrow x' \in V'\)

          \(-x \in V\) --- противоположный элемент к \(x\)

          \(-x' \in V\) --- противоположный элемент к \(x'\)

          \(\Rightarrow -x \longleftrightarrow -x'\)

          Доказательство

          \(\forall \lambda \in K: \lambda x \longleftrightarrow \lambda x'\)

          Пусть \(\lambda = -1\), тогда \(-x = -1 \cdot x \longleftrightarrow -1 \cdot x' = -x'\)
          \hfill Q.E.D.

    \item
          \(x_1, \ldots, x_m \in V; x_1' \ldots x_m' \in V'\)

          \(\forall k = 1 \ldots m: x_k \longleftrightarrow x_k'\)

          \(\Rightarrow \sum\limits_{k = 1}^{m} \alpha_k x_k \in V \longleftrightarrow \sum\limits_{k = 1}^{m} \alpha_k x_k' \in V'\)

          Доказательство

          По методу математической индукции
          \hfill Q.E.D.

    \item \(\underbracket{x_1, \ldots
              , x_m}_{\text{линейно независимы}} \in V \longleftrightarrow \underbracket{x_1', \ldots, x_m'}_{\text{линейно независимы}} \in V'\)

          Доказательство

          \(\alpha_k \in K\)

          \(\sum\limits_{k = 1}^{m} \alpha_k x_k = 0 \longleftrightarrow \sum\limits_{k = 1}^{m} \alpha_k x_k' = 0'\)

          т.к. \(\sum\limits_{k = 1}^{m} \alpha_k x_k \longleftrightarrow \sum\limits_{k = 1}^{m} \alpha_k x_k'\) (3 свойство) и \(0 \in V \longleftrightarrow 0' \in V'\) (1 свойство)

          \(\underbracket{x_1, \ldots
              , x_m}_{\text{линейно независимы}} \in V \Leftrightarrow \forall k = 1 \ldots m: \alpha_k = 0 \Leftrightarrow \underbracket{x_1', \ldots
              , x_m'}_{\text{линейно независимы}} \in V'\)
          \hfill Q.E.D.

    \item \(\underbracket{x_1, \ldots
              , x_m}_{\text{порождающая система}} \in V \longleftrightarrow \underbracket{x_1', \ldots, x_m'}_{\text{порождающая система}} \in V'\)

          \(x_1, \ldots, x_m \in V\) --- порождающая система \(\Leftrightarrow \forall x \in V: x = \sum\limits_{k = 1}^{m} \alpha_k x_k\)

          \(\forall x \in V: x = \sum\limits_{k = 1}^{m} \alpha_k x_k \longleftrightarrow \forall x' \in V': x' = \sum\limits_{k = 1}^{m} \alpha_k x_k'\)

          т.к. \(\sum\limits_{k = 1}^{m} \alpha_k x_k \longleftrightarrow \sum\limits_{k = 1}^{m} \alpha_k x_k'\) (3 свойство) и \(x \longleftrightarrow x'\)

          \(\forall x' \in V': x' = \sum\limits_{k = 1}^{m} \alpha_k x_k' \Leftrightarrow x_1', \ldots, x_m'\) --- порождающая система
          \hfill Q.E.D.

    \item \(\underbracket{e_1, \ldots, e_n}_{\text{базис } V} \longleftrightarrow \underbracket{e_1', \ldots, e_n'}_{\text{базис } V'}\)

          Доказательство

          Из свойств 4 и 5 мы знаем, что если система векторов линейно независима и порождающая, то есть это базис
          \hfill Q.E.D.
\end{enumerate}


\subsubsection{Координатный изоморфизм.Теорема об изоморфизме конечномерных про-
странств, следствие.}

\thmm{Теорема}

\(V_1, V_2\) --- линейные пространства над полем \(K\)

\(V_1 \cong V_2 \Leftrightarrow \dim V_1 = \dim V_2\)

\begin{enumerate}
    \item[] \prooff{}
    
\fbox{\(\Leftarrow\)}
\(\dim V_1 = \dim V_2 \Rightarrow e_1, \ldots, e_n\) --- базис в \(V_1\) и \(e_1', \ldots, e_n'\) --- базис в \(V_2\)

Построим изоморфизм из \(V_1\) в \(V_2\)


\[
    x = \sum\limits_{k = 1}^{n} x_i e_i \in V_1 \longleftrightarrow x = \begin{pmatrix}
        x_{1}  \\
        \vdots \\
        x_{n}
    \end{pmatrix} \in K^n \longleftrightarrow x' = \sum\limits_{k = 1}^{n} x_i e_i' \in V_2
\]

\[
    x \in V_1 \xleftrightarrow{\substack{\text{координатный} \\ \text{изоморфизм}}} x \in K^n \xleftrightarrow{\substack{\text{координатный} \\ \text{изоморфизм}}} x' \in V_2
\]

Проверим линейность \(\forall \lambda \in K\)

\[
    x + \lambda y \longleftrightarrow \sum\limits_{k = 1}^{n} x_i e_i + \lambda \sum\limits_{k = 1}^{n} y_i e_i = e_i (\sum\limits_{k = 1}^{n} x_i + \lambda \sum\limits_{k = 1}^{n} y_i) \longleftrightarrow \begin{pmatrix}
        x_{1} + \lambda y_1 \\
        \vdots              \\
        x_{n} + \lambda y_n
    \end{pmatrix} \longleftrightarrow
\]

\[
    \longleftrightarrow \sum\limits_{k = 1}^{n} x_i e_i' + \lambda \sum\limits_{k = 1}^{n} y_i e_i' = e_i' (\sum\limits_{k = 1}^{n} x_i + \lambda \sum\limits_{k = 1}^{n} y_i) \longleftrightarrow x' + \lambda y'
\]

\[
    x + \lambda y \longleftrightarrow x' + \lambda y'
\]

Биекция сохраняет свойство линейности \(\Rightarrow\) изоморфизм

\fbox{\(\Rightarrow\)}
Если \(V_1 \cong V_2\), то из 6 свойства изоморфизма мы знаем, что существует биекция между базисами этих систем \(\Rightarrow \dim V_1 = \dim V_2\)

\hfill Q.E.D.
\end{enumerate}
\textbf{Следствие.} Изоморфность разбивает множество линейных пространств на классы эквивалентности с равной размерностью.

\subsection{Теорема о лин. подпространстве, ранг, база.}
\subsubsection{Теорема о лин. подпространстве.}
\thmm{Теорема} (критерий линейного подпространства)

\(L\) --- линейное подпространство \(V \Leftrightarrow \forall x, y \in L \subset V \ \forall \lambda \in K: x + \lambda y \in L\)

(\(L\) замкнуто относительно \(+, \cdot\))


\begin{enumerate}
    \item \prooff{}
    \fbox{\(\Rightarrow\)}
т.к. \(L \subset V\) и выполняются 1-8 аксиомы

\fbox{\(\Leftarrow\)}
т.к. \(L \subset V\) выполнены все аксиомы кроме 3 и 4

Пусть \(x \in L \subset V\), тогда \(x + (-1) \cdot x \in L \Rightarrow o \in L \Rightarrow \exists\) нейтральный элемент в \(L\)

Пусть \(x = 0 \in L, y \in L \Rightarrow 0 + (-1) \cdot y = -y \in L \Rightarrow \exists\) противоположный элемент

\(\Rightarrow\) для \(L\) выполнены 1-8 аксиомы линейного пространства

\hfill Q.E.D.
\end{enumerate}

\subsubsection{База, ранг системы векторов.}
\deff{Ранг системы векторов} \(\defLeftrightarrow \dim (\spann (v_1, \ldots, v_m)) = r = \rg (v_1, \ldots, v_m)\)

\(r\) --- число \(\max\) линейно независимых векторов в \(L = \spann (v_1, \ldots, v_m)\)

База --- сделай прополку, получишь базис span.  


\subsubsection{Теорема о ранге.}

Элементарные преобразования системы векторов:

\begin{enumerate}
    \item удаление/добавление нулевого вектора

    \item изменение порядка векторов

    \item замена любого векторов на него де, умноженный на скаляр (\(\lambda \in K, \lambda \neq 0: v_j \rightarrow \lambda v_j\))

    \item замена любого из векторов на его сумму с любым другим вектором системы (\(v_j \rightarrow v_j + v_k\))
\end{enumerate}

\textbf{Теорема}

\(\rg (v_1, \ldots, v_m)\) не меняется при элементарных преобразованиях

Доказательство очев.

\subsection{Пересечение и сумма линейных подпространств. Формула Грассмана.}
\(L_1, L_2 \in V\) --- линейные подпространства пространства \(V\)

\(L_1 + L_2 = \{ x_1 + x_2 \in V : x_1 \in L_1, x_2 \in L_2 \}\)

\(L_1 \cap L_2 = \{ x \in V : x \in L_1, x \in L_2 \}\)

\textbf{Теорема} (формула Грассмана)

\(L_1, L_2 \in V\) --- линейные подпространства пространства \(V\)

\(\dim (L_1 + L_2) = \dim (L_1) + \dim (L_2) - \dim (L_1 \cap L_2)\)

\textbf{Доказательство}



\begin{enumerate}
    \item \(\dim (L_1 \cap L_2) \neq 0\)

          \(L_1 \cap L_2 \neq \{0\}\). Возьмем базис. Дополним $L_1,L_2$ до базиса базисом пересечения. А дальше просто блаблабла

    \item \(\dim(L_1 \cap L_2) = 0\)

          \(L_1 \cap L_2 = \{0\}\). Тривиально.
\end{enumerate}

\subsection{Прямая сумма линейных подпространств. Теорема об эквивалентных условиях прямой суммы, следствие.}
\(L_1, \ldots, L_m \subset V\) называются дизъюнктными, если \(x_1 + \cdots + x_m = 0\), где \(x_i \in L_i, i = 1 \ldots m \Leftrightarrow \forall i = 1 \ldots m: x_i = 0\)

\(L_1 + \cdots + L_m\) называется прямой суммой, если \(L_1, \ldots, L_m\) --- дизъюнктны

\(L_1 \oplus L_2 \oplus \ldots \oplus L_m\) --- прямая сумма линейного подпространства

\textbf{Теорема}

\[
    L = L_1 + \cdots + L_m = \sum\limits_{k = 1}^{m} L_k, L_k \subset V
\]

\[
    L = \bigoplus_{k = 1}^{n} L_k \Leftrightarrow \text{выполнению любого из 3-х утверждений}
\]

\begin{enumerate}
    \item \(\forall j = 1 \ldots m: L_j \cap \sum\limits_{k \neq j} L_k = \{0\}\)

    \item базис \(L = \) объединение базисов \(L_k\)

    \item \(\forall x \in L: \exists! x_k \in L_k: x = \sum\limits x_k\) (единственность представления суммы)
\end{enumerate}

\begin{enumerate}
    \item[]\prooff{}
    Докажем первое условие. Сначала докажем его необходимость для дизъюнктности (то есть что из неё следует условие). Мы знаем, что $v_1+v_2+\cdots+v_m=0$ возможно только если каждый из векторов --- $0$. Рассмотрим $v\in L_i\cap\sum\limits_{\substack{j=1\\j\neq i}}^mL_j$. Он, как несложно заметить, лежит в $L_i$, поэтому может быть записан как $v_i$. То есть $v\in\sum\limits_{\substack{j=1\\j\neq i}}^mL_j$, что значит, что его можно записать как сумму $\sum\limits_{\substack{j=1\\j\neq i}}^mv_j$. А это значит, что $-v_i+\sum\limits_{\substack{j=1\\j\neq i}}^mv_j=0$. По причине дизъюнктности, все слагаемые тут --- $0$. А значит $-v_i=0\Rightarrow v=0$. То есть любой $v\in L_i\cap\sum\limits_{\substack{j=1\\j\neq i}}^mL_j$ является $0$, что и требовалось доказать.\\
    
            Теперь докажем достаточность первого условия. Мы знаем, что $\forall i\in[1:m]~L_1\cap\sum\limits_{\substack{j=1\\j\neq i}}^mL_j=\{0\}$. Хочется доказать, что $v_1+v_2+\cdots+v_m=0\Leftrightarrow\forall i\in[1:m]~v_i=0$. Заметим, что $v_1+v_2+\cdots+v_m=0\Leftrightarrow\sum\limits_{\substack{j=1\\j\neq i}}^mx_j=-x_i$. Правая часть лежит в $\sum\limits_{\substack{j=1\\j\neq i}}^mL_j$, а левая --- в $L_i$. Это значит, что обе части лежат в их пересечении, а там лежит только $0$. Значит $v_i=0$. То же самое можно провести для любого $i$, получив, что все $v_i$ --- нули. Что и требовалось доказать.\\

           Дальше можно доказать, что первое условие равносильно второму, а второе равносильно третьему.
\end{enumerate}
\textbf{Следствие}

\(L = L_1 \oplus \ldots \oplus L_m \Leftrightarrow \dim L = \sum\limits_{i = 1}^{m} \dim L_i\)

\textbf{Доказательство}

по Грассману и мат. индукции
\hfill Q.E.D.

\[
    V = \bigoplus\limits_{i = 1}^{m} L_i \Rightarrow \forall x \in V: \exists! x_i \in L_i: x = \sum\limits_{i = 1}^{m} x_i
\]

\(x_i\) --- проекция элемента \(x\) на подпространство \(L_i\) параллельно \(\sum\limits_{j \neq i} L_j\)


\subsection{Многообразие и все о них.}
\deff{Линейным} (аффинным) многообразием называется множество точек пространства \(V: D = \{x \in V: x = x_0 + l, l \in L\}\), где \(L \subset V, x_0 \in V\) (сдвинутое линейное подпространство)

Размерность линейного многообразия\(\defLeftrightarrow \dim D = \dim L\)

\textbf{Теорема}

\(P_1 = x_1 + L_1; P_2 = x_2 + L_2\), где \(L_1, L_2 \subset V\) --- линейные подпространства, \(x_1, x_2 \in V\)

\[
    P_1 = P_2 \Leftrightarrow
    \begin{cases*}
        L_1 = L_2 = L \\
        x_1 - x_2 \in L
    \end{cases*}
\]

Доказательство очевидно. Справа налево тривиаьно, слева направо рассмотрите $p_1 = x_1 +l_1 = x_2 + l_2$ И посмотрите на $p_1=x_1$ или $p_2=x_2$

\textbf{Следствие}

\(P = X_0 + L\)

\(\forall x \in P \Rightarrow P_x = x + L = P\)

%todo: доделать линейное многообразие
\begin{enumerate}
    \item[] \prooff{} \begin{enumerate}
    \item \(L = L\)

    \item \(x - x_0 \in L\)
\end{enumerate}

\hfill Q.E.D
\end{enumerate}


\subsection{Фактор пространство лин. пространства}

Пусть у нас есть линейное подпространство $L$. Тогда отношение $x\sim y\Leftrightarrow x-y\in L$ является отношением эквивалентности, для любых векторов из $V$.

  \deff{Факторпространство} пространства $V$ {по модулю} линейного {подпространства} $L$ $V\big|_L$ --- это фактормножество $V$ по отношению эквивалентности $\sim$ из предыдущего утверждения.

   \thmm{Теорема}
   $V\big|_L$ состоит из линейных многообразий на $L$.
        \begin{enumerate}
            \item[] \prooff{} 
            Если $x-y\in L$, то линейные многообразия $x+L$ и $y+L$ по одной из теорем ранее совпадают. То есть эквивалентные элементы порождают одинаковые многообразия.
        \end{enumerate}

       
\thmm{Теорема} 
        
        $\dim V\big|_L=\dim V-\dim L$.
        \begin{enumerate}
            \item[] \prooff{} 
            Пусть $\{e_1;e_2;\ldots;e_m\}$ --- базис $L$. Дополним его до базиса $V$ векторами $\{f_1;f_2;\ldots;f_{n-m}\}$. Хочется доказать, что $\{f_1+L;f_2+L;\cdots;f_{n-m}+L\}$ --- базис $V\big|_L$.\\
            Докажем, что эта система порождающая. Нужно породить $v+L$. $v$ раскладывается по базису $\{e_1;e_2;\ldots;e_m;f_1;f_2;\ldots;f_{n-m}\}$ как $v=\sum\limits_{i=1}^m\alpha_ie_i+\sum\limits_{i=1}^{n-m}\beta_if_i$. Первая сумма лежит в $L$, то есть её можно выкинуть, многообразие останется таким же. А значит $v+L$ можно представить как $\sum\limits_{i=1}^{n-m}\beta_i(f_i+L)$, ведь по определению суммы многообразий это $\left(\sum\limits_{i=1}^{n-m}\beta_if_i\right)+L$.\\
            Теперь докажем линейную независимость. Рассмотрим нулевую линейную комбинацию $\sum\limits_{i=1}^{n-m}\beta_i(f_i+L)$. Она, как мы уже знаем, равна $\left(\sum\limits_{i=1}^{n-m}\beta_if_i\right)+L$. Это должно быть равно нейтральному элементу (то есть $L$). Когда эти линейные многообразия равны? Когда $\sum\limits_{i=1}^{n-m}\beta_if_i\in L$. То есть $\sum\limits_{i=1}^{n-m}\beta_if_i-\sum\limits_{i=1}^{m}\alpha_ie_i=0$. Но это же линейная комбинация векторов подсистемы $\{e_1;e_2;\ldots;e_m;f_1;f_2;\ldots;f_{n-m}\}$, а значит она линейно независима. А значит $\forall i\in[1:n-m]~\beta_i=0$, что значит, что линейная комбинация $\sum\limits_{i=1}^{n-m}\beta_i(f_i+L)$ тривиальна.
        \end{enumerate}
        
